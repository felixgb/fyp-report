\documentclass[xcolor=dvipsnames, aspectratio=169]{beamer}
\usepackage{helvet}
\usepackage{sourcecodepro}
\usepackage{listings}
\usepackage{xcolor}

\definecolor{commentcolor}{rgb}{0.51,0.58,0.57}
\definecolor{keywordcolor}{rgb}{0.16,0.63,0.59}
\definecolor{stringcolor}{rgb}{0.34,0.43,0.76}

\beamertemplatenavigationsymbolsempty%

\setbeamercolor{frametitle}{fg=Magenta,bg=Cyan!50}
\setbeamercolor{title}{fg=Magenta,bg=Cyan!50}
\setbeamertemplate{itemize item}{\color{Cyan}$\blacksquare$}
\setbeamertemplate{itemize subitem}{\color{Cyan}$\blacktriangleright$}

\lstset{%
    basicstyle=\tiny\ttfamily,
    captionpos=b,
    xleftmargin=0.5cm,
    showstringspaces=false,
    commentstyle=\color{commentcolor},
    keywordstyle=\color{keywordcolor},
    stringstyle=\color{stringcolor}
}

\begin{document}

\title{Higher Kinded Types and Lifetimes}
\author{Felix Bowman}

\date{\today}

\frame{\titlepage}

\begin{frame}
    \frametitle{This Project}
    \begin{itemize}
        \item The main focus was the implementation of the Type Checker
        \item Language Design considerations
        \item Tests!
    \end{itemize}
\end{frame}

\begin{frame}
    \frametitle{Types}
    Types are for more than just simple errors
    \begin{itemize}
        \item Polymorphism
        \item Memory Safety
    \end{itemize}
\end{frame}

\begin{frame}[fragile]
    \frametitle{Parametric Polymorphism}
    Reduces code duplication, parametrizing similar code. Generics in Java are
    an example:
    \begin{minipage}{0.45\textwidth}
        \begin{lstlisting}[language=Java]
class List<T> {
    public T getAt(int index) {
        // get object of type T
    }

    public boolean add(T e) {
        // add an element of type T to the list
    }
}
        \end{lstlisting}
    \end{minipage}%
    \begin{minipage}{0.45\textwidth}
        \begin{lstlisting}[language=Java]
class List {
    public Object getAt(int index) {
        // get any sort of object!
        // who knows what it will be.
    }

    public boolean add(Object e) {
        // add an object to list
    }
}
        \end{lstlisting}
    \end{minipage}

    This can make a language much nicer to use, and plays nicely with lots of
    functional concepts.
\end{frame}

\begin{frame}[fragile]
    \frametitle{Higher Kinded Types}
    Allow parametrization with type constructors as well as concrete types.
    \begin{lstlisting}[language=Java, basicstyle=\footnotesize\ttfamily]
interface Mappable<F<A>> {
    F<B> map(Function<A, B> f);
}
    \end{lstlisting}
    (Note: not real Java)
    This is just one example, but lots of constructs in programming languages
    make good use of this idea.
\end{frame}

\begin{frame}
    \frametitle{Memory Management}
    It's pretty useful to be able to store and reference areas of memory.  Heap
    memory allocation and deallocation is problematic though.
    \begin{itemize}
        \item Manual memory management by the programmer is a notorious source
            of bugs
        \item Garbage collection is slow
    \end{itemize}
\end{frame}

\begin{frame}
    \textit{Resource Acquisition Is Initialization}, or RAII, ties dynamically
    allocated heap memory to the lexical lifetime of an object. Memory is freed
    when a variable goes out of scope.
    \begin{itemize}
        \item If RAIIed heap memory cannot be aliased then lots of the problems
            associated with manual dynamic allocation (dangling references,
            memory leaks, double free) are solved. So only allow one mutable
            reference to heap memory at any one point.
        \item Immutable references can still be created, as long as they do
            not live longer then the memory that they reference.
    \end{itemize}
\end{frame}

\begin{frame}[fragile]
    This idea can be enforced by a \textbf{Borrow Checker} and
    \textbf{Lifetime Checker} at compile time as part of semantic analysis.
\end{frame}

\begin{frame}[fragile]
    \frametitle{Examples}
    \begin{lstlisting}[basicstyle=\footnotesize\ttfamily]
{
    let x = Thing::new();       // Allocate some heap memory
    let y = x;                  // Alias the value in x
    println("()", x);           // x has moved into y, so can 
                                // no longer be referenced.
                                // This is a compile time error.
}
    \end{lstlisting}
\end{frame}

\begin{frame}[fragile]
    \begin{lstlisting}[basicstyle=\footnotesize\ttfamily]
fn foo(val1: Bar, val2: Bar) -> (u32, Bar, Bar) {
    // do something with val1 and val2.
    // pass back ownership, so they can still be used by the caller.
    (123, val1, val2)   // This is pretty clunky.
}

// & denotes a borrow value. Borrows are immutable by default.
fn foo(val1: &Bar, val2: &Bar) -> u32 {
    // do something with val1 and val2, but cannot mutate them.
    123
}
    \end{lstlisting}
\end{frame}

\begin{frame}[fragile]
    \frametitle{Language Design}
    Extension of the Lambda Calculus, with higher kinded types and dynamic
    memory allocation.
\end{frame}

\begin{frame}[fragile]
    \frametitle{Implementation}
    \begin{itemize}
        \item Haskell
        \item \texttt{Alex} for lexing, \texttt{Happy} for parsing
        \item \texttt{HUnit} for later language tests
        \item Tests are mostly langauge source code.
    \end{itemize}
    Phases are split into parser, lifetime checker, borrow checker, 
    type checker with higher kinds, and evaluation in some cases
\end{frame}

\begin{frame}
    \textbf{Questions?}
\end{frame}

\end{document}
