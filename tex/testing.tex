\chapter{Key Tests}\label{sec:testing}

This section describes key test programs that demonstrate programming
with higher kinded types and lifetimes. The syntax the language is described
in Chapter~\ref{sec:formal}. Additional constructs have been added for clarity,
such as allowing a term to be defined with a name.

\section{Simple Programs}
The programs in this section serve as an introduction to the language syntax
and style, without introducing more advanced features. Some of the examples are
polymorphic in their type.

\subsection{Polymorphic Identity Function}
This example shows the polymorphic identity function:

\begin{lstlisting}[mathescape, caption={Identity function}, label={lst:id}]
id: (X, 'a) $\rightarrow$ (X, 'a)
id = $\lambda$X:: * . $\lambda$'a . $\lambda$x: (X, 'a) . x
\end{lstlisting}

Above, the type of the function is give. All function parameters are associated
with a lifetime as well as a type. This is what is meant by the pair
\lstinline[mathescape]{(X, 'a)}.  Here \lstinline{X} is a type parameter which
is introduced starting in the scope of the function
\lstinline[mathescape]{$\lambda$X::* .}. The while \lstinline{id} term must be
applied to type first, as the type lambda expression is outermost. The type
that the expression is applied to must be of kind \lstinline{*}, like
\lstinline{Int} or \lstinline[mathescape]{X $\rightarrow$ X}. For example,
\lstinline{id [Int]} is type correct. 

The term \lstinline[mathescape]{$\lambda$'a .}\ldots is a lifetime lambda.
This term must be applied to some lifetime expression, such as another lifetime
variable or a static lifetime.

\subsection{Multiple Lifetimes and References}
More concepts are introduced in this example:

\begin{lstlisting}[mathescape, caption={Addition of two references}, label={lst:add}]
add: (&Int, 'a) $\rightarrow$ (&Int, 'b) $\rightarrow$ (Int, min('a, 'b))
add = $\lambda$'a . $\lambda$'b . $\lambda$x: (&Int, 'a) . $\lambda$y: (&Int, 'b) . !x + !y
\end{lstlisting}

Here the dereference operator is used to get the values in the locations held
by \lstinline{x} and \lstinline{y}. Both of these variables have the type of
references of \lstinline{Int}, demonstrated in the type, \lstinline{&Int}.
These values have an associated lifetime, or the length that they will exist in
memory for. The lifetime of the whole resulting expression is the either
\lstinline{'a} or \lstinline{'b}, whichever exists for the shortest length of
time. In other words, whichever lifetime is at the top of the stack. \Note{not
very descriptive here}

It is quite useful to have multiple lifetimes bound here. \lstinline{x} may exist
for a different length of time then \lstinline{y}. A common real world example
would be searching for some value in a larger data structure; the value to
search for would only need to exist for the length of the search function, whereas the
structure being search might need to exist for a much longer time.

\subsection{Higher Order Functions}
This example shows functions being used as first class values in the language,
which is not very interesting in itself, but the type signatures of the
functions are of note. In the simply typed lambda calculus, functions may have type
\lstinline[mathescape]{X $\rightarrow$ X}. Here, functions must annotate the
variable that they bind with a lifetime as well as a type. Hence, a function
might have the signature \lstinline[mathescape]{(X, 'a) $\rightarrow$ (X, 'a)},
meaning that a function takes a value of type \lstinline{X} that lives for
\lstinline{'a} and return a value of the same type and lifetime. 

\begin{lstlisting}[mathescape, caption={Function composition}, label={lst:compose}]
compose: ((B, 'a) $\rightarrow$ (C, 'a)) $\rightarrow$ ((A, 'a) $\rightarrow$ (B, 'a)) $\rightarrow$ (A, 'a) $\rightarrow$ (C, 'a)
compose = $\lambda$A:: * . $\lambda$B:: * . $\lambda$C:: * .
     $\lambda$'a .
        $\lambda$f: (B, 'a) $\rightarrow$ (C, 'a) .
            $\lambda$g: (A, 'a) $\rightarrow$ (C, 'a) .
                $\lambda$x: (A, 'a) . f (g x)
\end{lstlisting}

The signatures of \lstinline{f} and \lstinline{g} show that they may return
values with different types, but the same lifetimes. In fact, all values in
this version of \lstinline{compose} must live for the same length of time.

%\section{Higher Order

\section{Memory safety}
The intention of this section is to show how programs written in an unsafe
language, like C, can be transformed into a language with memory safety. 
Specifically, problems in C programs will be highlighted and an equivalent
program will be shown that does not compile.

dangle example, calc example that does not typecheck, explain why
\section{Basic tests}
Here basic aspects such as definition

\begin{lstlisting}[mathescape, caption={Double Function application}, label={lst:double}]
double: 
double = $\lambda$X:: * . $\lambda$a' . $\lambda$f: (X, 'a) $\rightarrow$ (X, 'a) . $\lambda$x:(X, 'a) . f (f x)
\end{lstlisting}

\begin{lstlisting}[mathescape, caption={Quadruple application function using double}, label={lst:quad}]
quad: 
quad = $\lambda$X:: * . $\lambda$'a . double [X $\rightarrow$ X] a' (double [X] 'a)
\end{lstlisting}
