\chapter{Parametric Polymorphism}\label{sec:poly}
This section introduces Parametric Polymorphism, which is found in many
programming languages. It is used to make programming languages more expressive
while attempting to prevent errors from occurring at runtime.  Polymorphism is
one way of sticking to a core programming principle: Functionality in a program
should only be implemented in one place in the source code. If different
functions are meant to do very similar things, it is generally beneficial to
combine them into one by parametrizing sections that differ. This allows code
to be updated, maintained, and deleted more easily and so becomes very
necessary in large codebases.  This parametrizing of areas of code is known as
abstraction. Abstraction at the levels of values is very common in the form of
functions; parametric polymorphism allows for the abstraction at the level of
types.

The rest of this chapter attempts to explain how polymorphism is useful, and
how it is used in languages like Java to reduce code duplication and increase
expressiveness. Generics, or parametric polymorphism in Java, is used
extensively in Java's collections framework, which provides very general data
structures for organizing data in programs. These structures are practical to
implement and use partly because of Generics.

Finally, Functors are introduced as an concept that we would like to be able to
represent and use in a programming language, but cannot do so in languages with
the expressive power of Java. Higher Kinded Types are introduced to provide
very generic solutions to programming problems, and form the basis of a large
part of this project.

\section{A World Without Generics}\label{sec:nopoly}
In software engineering one of the principles to keep in mind when attempting
to write software is Don't Repeat Yourself, known as the DRY principle. If lots
of similar behaviour exists in different places in a codebase, then a change to
this behaviour will result in the code being modified in several places.  Such
changes to behaviour are common; and not applying changes to one area while
applying them to another area can be a source of bugs. There are other reasons
not repeat oneself in a codebase as well. Less time is spent writing code if a
function only has to be written once, and lots of similar code can distract
from the meaning of a program.

As an example, consider writing a list class to store some strings in a
language like Java. The class may have an implementation that looks something
like this:

\begin{lstlisting}[caption=String List in Java, language=Java, label={lst:javastringlist}]
class StringList {

    public boolean add(String e) {
        // implementation...
    }

    public String getAt(int index) {
        // Get the string at position index...
    }

    public int size() {
    }

    // other sensible methods that you might want in a list...
}
\end{lstlisting}

This seems all very well and good. However, at a different part of the program
we might want to have a list of \lstinline{Booleans}:

\begin{lstlisting}[caption=A very similar class, language=Java, label={lst:javaboolist}]
class BoolList {

    public boolean add(Boolean e) {
        // add a boolean to the list
    }

    public Boolean getAt(int index) {
        // Get the bool at position index...
    }

    public int size() {
    }
}
\end{lstlisting}

The similarity between these classes is obvious: the methods that they
implement are very similar except for the types in method signatures.
\lstinline{BoolList}'s \lstinline{getAt} methods returns a \lstinline{Boolean},
and \lstinline{StringList}'s \lstinline{getAt} returns a \lstinline{String}. It
is easy to see that the implementation for both of the \lstinline{getAt}
methods would be very similar.  Now imagine if there was a bug in the
implementation of this method. This bug would likely be present in both of the
implementations and so a programmer would have to fix both of these
implementations. It would be very possible for the programmer to forget or
ignore one of these methods, leading to inconsistent behavior in the program.
Even if the programmer did remember to change the codebase in two places, there
would still be twice as much code to change. This becomes more of a problem for
every new list type that the programmer wishes to implement.

One might be tempted to leverage Java's inheritance to solve this issue. A
general list class that stores Java's \lstinline{Object} class might be the
answer.  After all, \lstinline{String} and \lstinline{Boolean} both extend
\lstinline{Object}.  At first glance this seems like a sensible solution, any
changes to how lists would only have to be put into effect in this one class.
This would be a vast improvement on making changes to every type of list that
we implement. The class might look something like:

\begin{lstlisting}[caption=A more general approach., language=Java, label={lst:abstractlist}]
class List {

    public boolean add(Object e) {
        // add whatever you like to the list!
    }

    public Object getAt(int index) {
        // Get the whatever object is stored at index
    }

    public int size() {
    }
}
\end{lstlisting}

This class looks very similar to the types of list that were implemented in
Listings \ref{lst:javaboolist} and \ref{lst:javastringlist}. However, in the
signatures of the methods we use \lstinline{Object} so we can put whatever we
like in the list. The class could be used like this:

\begin{lstlisting}[nolol, language=Java, label={lst:usinglist}]
List list = new List();
list.add("This seems to work well!");
String firstElem = (String) list.getAt(0);
\end{lstlisting}

However, there could be a problem here. On the last line we get an element from
the list, using the \lstinline{getAt} method, which returns an
\lstinline{Object}.  An instance of type \lstinline{Object} is not very useful
because we probably want to call some class-specific methods on our element
that we pulled from the list. The example above gets around this by using
casting. Casting is a way of telling the Java compiler that one type should be
treated as another. Specifically it is often used to tell the compiler that the
class that is being casted from is actually a subtype of the class that is
being casted to. In this case it is used to tell the Java compiler that the
instance of \lstinline{Object} that is being returned from the
\lstinline{getAt} method is actually a \lstinline{String}. This is done so that
we can access all of the methods that the \lstinline{String} class provides,
and so this object can be used where ever an instance of the \lstinline{String}
class is expected.

Casting from \lstinline{Object} cannot be checked by the compiler. This snippet
shows what could potentially go wrong in this situation:

\begin{lstlisting}[nolol, language=Java, label={lst:casterr}]
List list = new List();
list.add(new Boolean(true));
String firstElem = (String) list.getAt(0);      // This line will compile
\end{lstlisting}

The above code compiles but will throw a \lstinline{ClassCastException} when
run because String is not a subclass of Boolean. The Java compiler cannot tell
from the above that the list object is holding \lstinline{Boolean} values, so
cannot warn the programmer that the above code does not make sense. Any safety
provided by the type system has been subverted and the code may be in
production somewhere before the obvious error is detected by a human, or worse,
throws a runtime exception. This is the kind of problem that Java's Generics
were intended to solve.

\section{Parametrizing Constructors}
Just as methods in Java may have formal parameters that abstract values out of
the method body, Generics allow types to be abstracted out of the
implementation of a class. When a class is constructed a type can be passed in
as an argument which can be referenced by the methods that the class provides.
The type that is passed into a class is referenced with a type variable, just
is arguments to methods are referenced by variables in the scope of that
method.  This has the benefit that code can be type-checked at compile time.
Casting, as demonstrated in Section~\ref{sec:nopoly}, can lead to run
time errors which can be difficult to find before they cause errors.

With Generics the list example from the previous section can be implemented
like this:

\begin{lstlisting}[caption={List class with Generics}, language=Java, label={lst:genericlist}]
class List<T> {
    public boolean add(T e) {
        // add something of type T to the list
    }

    public T getAt(int index) {
        // Get the object of type T stored at index
    }

    public int size() {
    }
}
\end{lstlisting}

Here the list is parametrized with some as yet unknown type, \lstinline{T}, in
the declaration \lstinline{class List<T>}. When an instance of this class is
created a type is passed into the constructor and the instance becomes a new,
specific type of list holding objects of that type. This list can be used like
this:

\begin{lstlisting}[nolol, language=Java, label={lst:listuse}]
List<String> list = new List<String>();
list.add("Much better!");
String firstElem = list.getAt(0); // No casting!
\end{lstlisting}

The \lstinline{String} type is passed into the \lstinline{List} type
constructor here. The \lstinline{T} type variable in
Listing~\ref{lst:genericlist} is instantiated with the \lstinline{String} type
and methods like \lstinline{getAt} return a \lstinline{String}. There is no
need for casting and the type correctness of this code can be checked at
compile time. If we tried to assign the result of the \lstinline{getAt} method
to a boolean variable here the program would not compile.

\section{More Generic Containers}
Any class that acts as a container for other types of object benefits from
being parametrized with the type of object that it holds. In the list example
earlier in the chapter it did not matter what kind of element the list was
meant to store, all that mattered was that there was a list of them. Most data
structures are this general, in that the type of element inside the structure
does not matter. This is where parametric polymorphism is most useful.

Binary trees are another example of a data structure that can store any type of
object. A node in a binary tree can either be empty, or hold an element and a
pointer to two other nodes. The element that is held in that tree can be of any
type-- a perfect opportunity to use generics. The implementation could look
something like this:

\begin{lstlisting}[caption={Binary tree with generics}, language=Java, label={lst:generictree}]
class Tree<T> {
    private Tree<T> left;       // pointer to the left branch
    private Tree<T> right;      // pointer to the right branch
    private T elem;             // Data that this node stores

    public void insert(T e) {
        // insert element into the tree
    }

    public boolean contains(T e) {
        // is e in this tree?
    }

    public int size() {
    }
}
\end{lstlisting}

Very similarly to the list implementation, the tree class must be instantiated
with some type, referred to as \lstinline{T} in the body of the class.
Both of these data structures act as generic containers.

This kind of abstraction, where types are parametrized in the body of classes,
is known as first-order polymorphism. This notion is formalized in language
called System F, described in Chapter~\ref{sec:formal}. However this report
attempts to explain higher-order polymorphism, a concept that cannot be
expressed in languages like Java. In order to understand why higher order
polymorphism is useful and why it cannot be expressed in Java it is important
to understand some additional concepts.

\section{\texttt{map} and Higher Order Functions}\label{sec:hof}
As mentioned, both the list in Listing \ref{lst:genericlist} and the tree in
Listing \ref{lst:generictree} can be thought of as containers that hold some
type of object. It is very common in programs to apply a function to every
element in collection. In Java, this is commonly expressed with a for-each
loop. For example, to get a new list of integers by adding one to every element
from an old list of integers one would write:

\begin{lstlisting}[nolol, language=Java, label={lst:forloop}]
List<Integer> intList = new ArrayList<>();
intList.add(1);
intList.add(2);
intList.add(3);

List<Integer> outList = new ArrayList<>();

for (Integer i : integerList) {
    outList.add(addOne(i));         // outList will contain [2, 3, 4]
}
\end{lstlisting}

Manipulating elements of a collection in this way is very common. However, all
we are doing here is applying a function to every element in this container.
It would be nice if one could apply the same kind of transformation to the
binary tree that we defined, without worrying details of tree traversal in this
section of program.

In functional programming this problem is solved by a specific higher-order
function, commonly called \lstinline{map}.  A higher-order function one that
takes another function as a parameter. \lstinline{map} as defined for our list
class would take a function as a parameter and apply that function to every
element in a list. The example from above could be written as:

\begin{lstlisting}[nolol, language=Java, label={lst:map}]
List<Integer> intList = new ArrayList<>();
intList.add(1);
intList.add(2);
intList.add(3);

List<Integer> outList = intList
    .stream()
    .map((x) -> addOne(x));     // Define an lambda function to pass to map
\end{lstlisting}

Here a lambda function is declared using Java 8 lambda syntax. The function
\lstinline{(x) -> addOne(x)} is declared as a without a name and passed as an
argument to the higher-order function, \lstinline{map}. The lambda function is
then applied to every element of the list and a new list is returned. 

The map function can be declared for any class that acts as a container holding
other values. It takes a function as a parameter and applies this function to
every the contents of that container. It would make sense to have a map
function for both lists and trees. It would also make sense to try define an
interface for containers that can be mapped over in this manner.

\section{\texttt{Functor} and Mapping over Containers}
Having an interface for containers that can be mapped over would be useful.
Any time we wanted to apply a function to every element in a container we could
pass a transforming function to the \lstinline{map} function which the
interface would provide and not have to worry about the implementation details
of iterating over every single element. Even if the type of collection were to
change, e.g.\ using a tree instead of a list, the transformation would hold
because both of the classes would implement the interface. This concept of a
mappable container is typically called Functor. We can try and write an
interface for this concept in Java:

\begin{lstlisting}[caption=An attempt to define Functor in Java., language=Java, label={lst:javaFunctor}]
interface Functor<A> {
    Functor<B> map(Function<A, B> f);
}
\end{lstlisting}

At first glance this interface seems to be acceptable. The interface takes a
type parameter, \lstinline{A}, which is the type of the element that the
container holds. One method is defined, \lstinline{map}, which takes a function \lstinline{f}
as a parameter. \lstinline{f} has a generic type; it takes an object
of type \lstinline{A} and returns an object of type \lstinline{B}. \lstinline{map}
itself returns an object than implements \lstinline{Functor} with elements of
type \lstinline{B}. 

However the return type of \lstinline{map} is not sufficient. It may return an
object of any class that implements \lstinline{Functor}, which will have the
correct type of elements \lstinline{B}, but will not be more specific than
that. Casting would be needed to use the collection again:

\begin{lstlisting}[nolol, language=Java, label={lst:map}]
List<Integer> intList = new ArrayList<>();
intList.add(1);
intList.add(2);
intList.add(3);

// Before casting, the result of map is just Functor<Integer>
List<Integer> outList = (List<Integer>) intList.map((x) -> addOne(x));
\end{lstlisting}

This results in a similar situation to the one described in the example of the
list that held only \lstinline{Object} in Listing~\ref{lst:abstractlist}.
Casting is needed to use the return value of \lstinline{map} as a list, so is
necessary if we wanted to use any list specific methods or pass it into a
function that expects a list as an argument. Again, casting is not an ideal
solution because it is susceptible to run time exceptions. In other words, this
Functor interface cannot be used in a type safe manner.

In order to define a generic Functor interface, where the return type of
\lstinline{map} is the same as the class that implements functor, we need to
reference that class that implements the interface inside the interface itself.
In other words, \lstinline{map} must know that it returns a \lstinline{List}
when the list class implements Functor, or a \lstinline{Tree} when the tree
class implements Functor. Hypothetically, this interface might look something
like:

\begin{lstlisting}[caption=Hypothetical type safe functor in Java, language=Java, label={lst:fakefunctor}]
interface Functor<F<A>> {
    F<B> map(Function<A, B> f);
}
\end{lstlisting}

Note that this is not valid Java. In the type parameter to this version of the
interface we define a generic class constructor, \lstinline{F}, that itself
take a single generic type parameter, \lstinline{A}. The map function then
returns the \textit{same} generic class, but parametrized with the result type
\lstinline{B}. In this manner we could constrain the map function to return the
same instance of Functor that it is called from.

\section{Higher Kinded Types}
%define type constructors?

In order to express concepts like the Functor described in Listing~\ref{lst:fakefunctor}
higher order polymorphism is needed. This means that we need the ability to pass
classes that themselves have type parameters, like \lstinline{List}, as type
arguments in classes. Type constructors can be represented by type variables in
the body of classes. This can be seen in the hypothetical Java functor example
above, where the polymorphic type constructor is represented by the type variable
\lstinline{F}. If Java allowed for type constructors to be represented as type
variables then the Functor syntax above might be one way of representing this.

Other generic concepts could be represented as well. Monads, a way of associating
data types with specific computations, are well represented with higher order
polymorphism. 

%
%\section{Higher Kinded Types}
%This section introduces Higher Kinded Types, also known as higher order
%polymorphism, and demonstrates the expressive power of this concept though
%examples.
%
%\subsection{Parametric Polymorphism}\label{sec:generics}
%First order parametric polymorphism, known as Generics in Java, are types that
%are parametrized over some other type. This kind of polymorphism allows for the
%definition and use of functions that behave uniformly over all types.  They can
%be used to write code that can be checked for safety at compile time. For
%example, without generics, a list could be used like:
%
%\begin{lstlisting}[caption=Runtime error that could be avoided, language=Java, label={lst:javaerr}]
%List l = new ArrayList();
%l.add("This is a string");
%Integer i = (Integer) l.get(0); // Run time error here
%\end{lstlisting}
%
%This is problematic because a list in Java can contain any type of object, but
%methods that the \lstinline{List} object provides must know about the type of
%object that the list contains. If those methods do not have a way of knowing
%what kind of object a list contains then calling the methods will not be type
%safe, as demonstrated in Listing~\ref{lst:javaerr}. The same code written with
%Java's Generics will produce a compile time error:
%
%\begin{lstlisting}[caption=Compile time error, language=Java, label={lst:generr}]
%List<String> l = new ArrayList<String>(); // Now the list has been parametrized with a type
%l.add("This is a string");
%Integer i = l.get(0);                     // Compile time error here.
%\end{lstlisting}
%
%Compile time errors are much more desirable then runtime errors because they
%can be caught and fixed predictably, unlike runtime errors.  The code in
%Listing~\ref{lst:generr} parametrizes the \lstinline{List} type with the
%\lstinline{String} type, and hence the last line produces a compile time error
%as the list's \lstinline{get} method returns a \lstinline{String}. 
%
%\subsubsection{Type Constructors}
%A type constructor is a feature of programming languages that builds new types
%from old ones. In Java, an example would be the \lstinline{List} type. The
%\lstinline{List} type cannot be used to describe some value in Java; all values
%must be a \lstinline{List} \textit{of} something, for example a
%\lstinline{List<String>}.  In other words, \lstinline{List} is a type
%constructor that must be parametrized with some base type like
%\lstinline{Integer}, \lstinline{String}, or even \lstinline{List<String>}. Once
%it has been applied to some base type, it becomes a new type;
%\lstinline{List<Integer>}, \lstinline{List<String>}, and
%\lstinline{List<List<String>>} are all separate types.
%
%Parametric polymorphism is an important addition to statically typed
%programming languages. First order polymorphism is modelled by a typed lambda
%calculus called System F, which is introduced in Section~\ref{sec:systemF}.
%However, Generics in Java and in lots of other programming languages leave
%something to be desired.  There are higher level, general, generic programming
%concepts that cannot be expressed with first order polymorphism. Generally,
%these systems are constrained in that type constructors may only be
%parametrized by concrete types, and not other type constructors. Higher order
%polymorphism is needed in order to neatly express some programming concepts.
%
%\subsection{Higher Order Polymorphism}
%As mentioned in Section~\ref{sec:generics}, first order parametric polymorphism
%can be very useful to programmers.  However, there are limitations. This
%section will attempt to demonstrate one shortfall of first order polymorphism
%and then show how the problems can be solved with higher order polymorphism.
%Functors are introduced as an example of one of the concepts that cannot be
%expressed neatly with first order polymorphism.
%
%\subsubsection{Functors}
%A Functor is important concept in modern programming languages. Most languages
%have constructs that can be thought of as Functors, such as lists, optional types,
%trees, and other constructs that can be mapped over. Anything that acts as a
%container for another type and provides a function for mapping a function over
%that contained type is a Functor. They are an important concept because they
%provide a common interface to working with any type that acts as a box for
%another type. This means that programs can be written more generically and is
%more open to changes and modifications.
%
%Some languages seek to reify the general concept of functors so that aspects of
%a program can be checked for correctness at compile time. However, expressing
%Functors in a generic way requires a more expressive type system, specifically
%a type system that incorporates higher order polymorphism. Here is an example
%of an attempt to define a general Functor interface in Java:
%
%\begin{lstlisting}[caption=An attempt to define Functor in Java, language=Java, label={lst:javaFunctor}]
%interface Functor<A> {
%    Functor<B> map(Function<A, B> f);
%}
%\end{lstlisting}
%
%
%\begin{lstlisting}[caption=Functor as defined in Haskell., language=Haskell, label={lst:haskellFunctor}]
%class Functor f where
%    fmap        :: (a -> b) -> f a -> f b
%\end{lstlisting}
%
%Listing~\ref{lst:haskellFunctor} shows how Functor can be defined in a Haskell,
%a language that does allow higher order polymorphism. Here, \lstinline{f} is a
%variable that references a type constructor. This class in Haskell specifies
%that anything that instanciates it must provide a single function,
%\lstinline{fmap}.  \lstinline{fmap} takes a function as a parameter that takes
%a value of type, \lstinline{a}, and returns a value of type \lstinline{b}. It
%then takes another argument of type \lstinline{f a}, or a type that the type
%constructor \lstinline{f} has been applied to. For example, it would take a
%value of type \lstinline{Maybe Int}, \lstinline{Maybe} being the type
%constructor \lstinline{f}. It would then return a value of type \lstinline{b},
%wrapped in the same type constructor \lstinline{Maybe}.
%
%The ability to use type constructors as first class in a programming language
%allows programs to be written in a more succinct, patterned, and generic manner.
%Examples like Functor show how a generic interface to work with constructs
%commonly found in software engineering can be achieved with the use of higher
%order polymorphism. 
%
%\subsubsection{Kinding}
%
%Higher order polymorphism can be thought of as the polymorphic lambda calculus
%`one level up', and as such the language at the level of types also needs to 
%have a system in place in order to prevent wrong expressions from being created.
%
%A system that allows for abstracting over type constructors needs some way
%of enforcing type constructors are not applied to types in a way that does not
%make sense, e.g.\ \lstinline{Integer Integer} does not make sense because the
%integer types does not take any type arguments in order to become a concrete
%type that can be inhabited by values. Alternately just \lstinline{List} cannot
%be instantiated with a value because it need one more type constructor, 
%the type that will be contained within that list.
%
%What is needed is essentially a type system for types. This is the notion of
%the kind of a type. Types can have the kind of $ * $, in which case
%they are concrete and can be inhabited by values, or they can have kind
%$ * \rightarrow * $, or a function from types to other types. Types like
%\lstinline{Integer} or \lstinline{List Integer} have kind $ * $ however
%type constructors like \lstinline{List} have kind 
%$ * \rightarrow * $ because they need to take one more concrete
%types before they can be used as a value. This system of kinding enforces that
%all types are well formed.
%
%\begin{figure}[H]\label{fig:kinds}
%    \centering
%    \begin{tabular}{l c p{3cm} r}
%        $ K $ & $ ::= $ & * &                   Concrete types \\
%      & $ | $ & $ K \rightarrow K $ &           Type functions from types to types \\
%    \end{tabular}
%    \caption{Syntax of Kinds}
%\end{figure}
%
%%%% unfuck this thing
%% 
%% \begin{figure}[h]\label{fig:kindexamples}
%%     \centering
%%     \begin{tikzpicture}[every node/.style={draw=none}]
%%         \node at (0, 0) (k){$ * $};
%%         \node at (13, 0) (k2k){$ * \rightarrow * $};
%%         \node at (17, 0) (k2k2k){$ * \rightarrow * \rightarrow * $};
%%         \node at (24, 0) (kk2k){$ (* \rightarrow *) \rightarrow * $};
%% 
%%         \node[below left=of k] (int) {\texttt{Int}};
%%         \node[below right=of int, xshift=-1cm] (listInt) {\texttt{List[Int]}};
%%         \node[above right=of listInt, xshift=-2cm] (pairInt) {\texttt{Pair[Int, Int]}};
%% 
%%         \node[below=of int, yshift=-2cm] (val) {\texttt{1}};
%%         \node[below right=of val, xshift=-2cm] (listVal) {\texttt{[1, 2, 3]}};
%%         \node[above right=of listVal, xshift=-2cm] (pairVal) {\texttt{(1, 2)}};
%% 
%%         \node (proper) [draw, fit= (k) (int) (listInt) (pairInt) (val) (listVal) (pairVal)] {};
%%         \node[yshift=3.0ex] at (proper.north) {\textit{Proper}};
%% 
%%         \draw[->] (val) -- (int);
%%         \draw[->] (listVal) -- (listInt);
%%         \draw[->] (pairVal) -- (pairInt);
%% 
%%         \draw[->] (int) -- (k);
%%         \draw[->] (listInt) -- (k);
%%         \draw[->] (pairInt) -- (k);
%% 
%%         %%% first order
%%         \node[below=of k2k] (list) {\texttt{List}};
%%         \node[below=of k2k2k] (pair) {\texttt{Pair}};
%% 
%%         \draw[->] (list) -- (k2k);
%%         \draw[->] (pair) -- (k2k2k);
%% 
%%         \node (first) [draw, fit= (k2k) (k2k2k) (list) (pair)] {};
%%         \node[yshift=3.0ex] at (first.north) {\textit{First Order}};
%% 
%%         %%% higher order
%%         \node[below=of kk2k] (functor) {\texttt{Functor}};
%% 
%%         \draw[->] (functor) -- (kk2k);
%% 
%%         \node (higher) [draw, fit= (functor) (kk2k)] {};
%%         \node[yshift=3.0ex] at (higher.north) {\textit{Higher Order}};
%% 
%%         \node[left=of k, xshift=-3cm] (kinds) {\textit{Kind}};
%%         \node[left=of int] (type) {\textit{Type}};
%%         \node[left=of val] (value) {\textit{Value}};
%%     \end{tikzpicture}
%%     \caption{Some types and their associated kinds.}
%% \end{figure}
%
%\section{Problems with Existing Memory Management Techniques}
%Most programming languages allow for dynamic memory management where the
%program can request portions of memory at run time and free that memory when it
%is no longer required. This operation can be performed manually as in the case
%of languages such as C or automatically in languages with a garbage collector
%such as Java.
%
%\subsection{Manual Memory Management: \lstinline{malloc} and \lstinline{free}}
%Some languages rely on manual instructions inserted by the programmer in order
%to allocate and deallocate regions of memory. Most implementations of the C
%programming language provide a group of library functions for this purpose,
%which include \lstinline{malloc} and \lstinline{free}. As mentioned, these
%constructs are insert manually by the programmer. This can lead to the issues
%described below.
%
%\subsubsection{Dangling References}\label{sec:dangle}
%Deallocating storage may lead to dangling references, where some reference 
%exists that points to memory that has been deallocated. This is nearly always
%unintentional, however it is an easy for a programmer to make this mistake when
%writing a program. Using the value of deallocated storage can lead to nefarious
%bugs and most languages consider using deallocated storage to
%be undefined behaviour.
%
%\lstinputlisting[
%    caption={A C program that leaves \texttt{p} pointing to deallocated 
%    memory.},
%    label={lst:cdangle}
%]{codeexamples/dangle.c}
%
%In the Rust programming language, references to values must live longer than
%the resource that they refer to~\cite{rust-borrowing}. The same program as the 
%C dangling pointer example (Listing~\ref{lst:cdangle}) written in Rust does not
%compile as it does not meet this restriction.
%
%\lstinputlisting[
%    caption={A Rust program that does not type check.}
%]{codeexamples/dangle.rs}
%
%\subsubsection{Memory Leaks}\label{sec:leak}
%Memory leaks occur when heap storage that is no longer usable is not
%deallocated. This can lead to software allocating more memory for itself and in
%the worst case can lead to a program consuming all available memory.
%
%\lstinputlisting[
%    caption={A C program which leaks memory.}
%]{codeexamples/leak.c}
%
%\subsection{Automatic Memory Management: Garbage Collection}
%Techniques have been invented in order to mitigate errors related to manual
%memory management. Garbage collection is one such method.  Garbage is a term
%for storage that has been allocated on the stack but is no longer reachable by
%the program, e.g.\ the storage is pointed to by a variable that has gone out of
%scope. Automatic garbage collection alleviates the programmer of explicitly
%managing memory. However there are several downsides to garbage collection. For
%example, the garbage collector will be invoked to collect unusable memory
%outside of the control of the programmer. When it is running it affects the
%execution of the program for an indeterminate amount of time. This hang is not
%acceptable in real--time systems.
%
%\section{A Solution in the Type System}
%Memory safety can still be achieved without the use of a garbage collectors.
%Program structure may be analysed during semantic checking and areas likely to
%cause issue can be identified. Here Rust's way of performing these checks is
%examined.
%
%\subsection{An Introduction to Rust's Model of Memory Management}
%Rust, a systems programming language~\cite{rust} is designed to provide memory
%safety with no runtime overhead. Rather than relying on a garbage collector to
%provide run-time memory safety, Rust relies on strong compile time checks.
%This system of compile time checks is referred to as \textit{ownership}, and is
%derived from affine type systems and unique pointers to memory
%locations\cite{rust-borrowing} \cite{levy2015ownership}.
%
%\subsubsection{Affine types}
%Affine types are a development of Linear type systems. Linear type systems
%ensure that every variable is used exactly once. In an affine type system, a
%variable may be used at most once, i.e.\ zero or one times. This has applications
%for managing resources in a program that do not persist the whole time that the
%program is running, e.g.\ file handles or heap allocated memory \cite{attapl}
%\cite{tovAffine}.
%
%\subsubsection{Ownership}\label{sec:regions}
%Values in Rust are bound to a single \textit{owner}, which is the variable to
%which it is bound. When the owning variable goes out of scope, the value is
%freed. For example:
%\begin{lstlisting}[nolol, label={lst:rustScope}]
%{
%    let x = 69;
%}
%\end{lstlisting}
%When \lstinline{x} goes out of scope the value associated with it will also be
%released from memory. Because there can only be a single owner associated with
%a value, aliasing a variable is not allowed. This prevents any two parts
%of a Rust program from accessing a section of heap memory at the same time.
%\begin{lstlisting}[nolol, label={lst:rustScope}]
%{
%    let x = Thing::new();
%    let y = x;
%    println("()", x);
%}
%\end{lstlisting}
%The above code does not compile because \lstinline{x} has been moved to
%\lstinline{y}, and there is an attempt to use \lstinline{x} in the print
%statement. Using a variable as a function argument will also make it not
%available for use again. If a function uses some value that has been passing in
%as a argument it must return that same value if the scope outside that
%function uses it.
%\begin{lstlisting}[nolol, label={lst:rustScope}]
%// value is comes from outside the scope of this function
%fn foo(value: Bar) -> Bar {
%    // do stuff with value, returning it and hence handing back ownership
%    value
%}
%\end{lstlisting}
%This is not practical, and gets even less practical once there are more
%arguments that must be passed back to the caller of the function. To this end
%Rust allows references to a value to be created. Creating a reference to a value
%in Rust is called borrowing. A reference to a value may be created and passed
%in to a function that uses it in some way, and that function does not need to
%thread the ownership of that variable back to the caller.
%\begin{lstlisting}[nolol, label={lst:rustScope}]
%// instead of passing value1 and value2 in by value, create references to values
%fn foo(value1: &Bar, value2: &Bar) -> u32 {
%    // do stuff with value1 and value2 and return the answer
%    123
%}
%\end{lstlisting}
%There are two kinds of borrows: immutable and mutable. There may be many
%immutable borrows of a resource at once or a single mutable borrow. This also
%prevents heap data from being accessed at the same time at different points in
%the program. The scope of borrow must always be shorter than that of the
%resource which it borrows. 
%
%This system where resources are freed after the single owner goes out of scope,
%as well as borrows always having shorter lifetimes than original resource
%mean that Rust can enforce the memory safety of programs. These concepts are
%built into the type system.
%
%\section{Representing Programs as Data Structures}
%talk about ASTs here
%
%\section{Type checking}
%talk about ast tree traversal
