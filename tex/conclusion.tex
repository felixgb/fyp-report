\chapter{Conclusion and Further Work}\label{sec:conclusion}
This section covers ideas that should be implemented in order to make the
lanagauge more useful to reason about how these type systems interact.

\section{Alternatives to Full Higher Kinded Types}
Full higher kinded types in Rust may not be necessary to get the benefits
commonly associated with this feature. An alternative proposition involves
adding type constructors to \textit{traits}, which would be known as
\textit{associated type constructors} \cite{rfc1598}. Rust already includes an
mechanism for associating types with traits, so expanding this feature to
include type constructors as well seems like a more natural extension than full
higher kinded types, as seen in other languages like Haskell. This section
investigates this idea and attempts to define some high level generic constructs
using only the proposed idea of Associated Type Constructors.

\subsection{Traits}
Haskell's typeclasses are key in expressing useful high level concepts like
Monad, Functor, and Applicative Functors \cite{consclasses}. Traits and
typeclasses are a way of providing a generic interface. In Rust, traits
programming language fill a very similar role to Haskell's type classes,
however traits may only describe concrete types, or types with kind
\lstinline{*}. If one were to bring higher order polymorphism to a language
like Rust then it would be essential to implement type classes, as they are
important in leveraging the benefits of higher order polymorphism anyway. There
might also be some difficulties in incorporating type classes into a language
with higher order polymorphism and Rust's model of memory management.

In essence, generic container objects must keep track of the lifetimes of the
objects that they contain.  In order to understand some of the problems with
implementing generic interfaces for containers (such as Functor, from
Chapter~\ref{sec:poly}) some additional concepts are needed. Examples are given
in Rust.

\subsection{Associated Types}
Associated types refer to types that may be defined in a trait, and used by
methods that are defined by that trait. This example from the Rust
documentation \cite{assoctypes} illustrates this concept.

Suppose that we wanted a trait to describe a graph, with generic node and edge
type:

\begin{lstlisting}[nolol]
trait Graph<N, E> {
    // ... useful graph methods defined here
    fn edges(&self, &N) -> Vec<E>;
}
\end{lstlisting}

The graph has been parametrized with the types \lstinline{N} for the type of 
nodes and \lstinline{E} for the type of edges. While this works well for operating
over a generic graph it is a bit cumbersome. For example, if we wanted to write
functions in Rust that operate over \lstinline{Graph} we would have to write:

\begin{lstlisting}[nolol]
fn distance<N, E, G: Graph<N, E>>(graph: &G, start: &N, end: &N) -> u32 {
    // ... distance implementation
}
\end{lstlisting}

There are quite a few type parameters in this function signature and it impedes
understanding. In particular, it is not clear that \lstinline{N} and
\lstinline{E} are associated with \lstinline{Graph}. \lstinline{distance} does
need to reference these types in its signature, though. To solve this issue
Rust introduces associated types. These are illustrated in a this revised 
\lstinline{Graph} example:

\begin{lstlisting}[nolol]
trait Graph {
    type N;     // type of nodes
    type E;     // type of edges

    fn edges(&self, &Self::N) -> Vec<Self::E>;  // Self references this trait
}
\end{lstlisting}

Note that the trait now has type definitions included in its definition. Graph
no longer takes type parameters, but does still reference the generic types
\lstinline{N} and \lstinline{E}. The signature of the \lstinline{distance}
function that operates over this new graph looks like:

\begin{lstlisting}[nolol]
fn distance<G: Graph>(graph: &G, start: &G::N, end: &G::N) -> u32 {
    // distance implementation
}
\end{lstlisting}

This function signature, as well as being more brief, also makes it very clear
that the node and edges types are associated with the graph type. 

Associated types define types that are part of an interface. Note that in the
current version of Rust, associated types may only be of kind \lstinline{*},
i.e.\ they may not be type constructors. Expanding associated types to include
type constructors may have some interesting benefits, commonly associated with
full higher kinded types \cite{niko}.

\subsection{Iterators and Generic Collections}
There are several motivations for including associated type constructors.

\subsubsection{Iterators}
Iterators allow programmers to traverse containers in a safe, generic manner.
They are an alternative accessing an collection with indexing operations and
are more suited to traversing collections that do not provide effective random
access, like lists or trees. Many types of collection may provide iterators and
so providing an iterator interface makes code more generic by not associating
operations on collections with the implementation of those operations.

In Rust, a trait that describes what operations an iterator should provide can
be written like:

\begin{lstlisting}[nolol]
trait Iterator {
    type Item;  // Associated type item describes what type the iterator will return
    fn next(&mut self) -> Option<Self::Item>;
}
\end{lstlisting}

The method defined here, \lstinline{next}, says that any class that implements
this trait must define how collection elements are accessed. Elements of type
\lstinline{Item} that are held by a collection are returned when there are more
elements to access. 

\subsubsection{Attempting to Define Generic Collections}
Making \lstinline{Iterator} a trait would allow code to be written that can
separate the logic of using elements of a collection from the implementation of
how those elements are accessed. To expand on this idea, it would be useful to
have an interface for all types of collections that provide a few key
operations. If this could be done, it would further separate implementation
details from logic of using collections. This would facilitate code reuse and
allow code to be swapped out or replaced more easily because the collection we
wanted to use would only have to conform to a given interface, or trait.  One
of the key methods that the trait would provide would be some way of accessing
elements of the collection in a generic way, probably using iterators.  It
might look something like:

\begin{lstlisting}[nolol]
trait Collection<T> {
    // Create an empty collection of type T
    fn empty() -> Self;

    // Add an element of type T to this collection. For linked lists, it might
    // prepend to the collection, for sets it might insert, etc
    fn add(&mut self, value: T);

    type Iter: Iterator<T>;     // An iterator for this collection would also
                                // be of type T, becuase it would reference
                                // a single element of this collection at one
                                // time

    // Get a reference to an iterator of this collection so you can do things
    // with the elements of this collection by calling the next method defined
    // in the Iterator interface
    fn iterate(&self) -> Self::Iter;
}
\end{lstlisting}

Now we can try and define a collection that implements all of the operations
described in this trait. Here's what a minimalistic linked list might look like
in Rust:

\begin{lstlisting}[nolol]
pub struct List<T> {
    cell: Option<Box<ListCell<T>>
}

struct ListCell<T> {
    value: T,
    next: List<T>
}
\end{lstlisting}

Some methods can be implemented for this collection:

\begin{lstlisting}[nolol]
impl<T> List<T> {
    pub fn new() -> List<T> {
        List { cell: None }
    }
    
    pub fn prepend(&mut self, value: T) {
        // Prepend an element to this collection...
    }
    
    pub fn iter<'iter>(&'iter self) -> ListIter<'iter, T> {
        ListIter { cursor: self }
    }
}

// Define a list iterator that does what an iterator should, specifically for
// lists.
pub struct ListIter<'iter, T> {
    cursor: &'iter List<T>
}

impl<'iter, T> Iterator for ListIter<'iter, T> {
    // Here the Item associated type is instantiated, from the Iterator trait
    // Note that the item is a reference to an element T, that lives for the
    // lifetime 'iter
    type Item = &'iter T;
    fn next(&mut self) -> Option<&'iter T> {
            // return a Option with type T, which is a reference to an element
            // in the collection, or the same type as Item
        }
    }
}
\end{lstlisting}

This implementation of an iterator for a list returns a reference to an
\lstinline{Item} from the list. \lstinline{Item} needs to have a lifetime
because any method that uses an element from this collection needs to know how
long the collection lives for. A reference to an element in the collection
could not be used after the collection itself has been deallocated. It does not
matter what the type of that element is though, only that it lives long enough.
\lstinline{ListIter} takes a lifetime parameter accordingly.

\lstinline{List} should implement \lstinline{Collection}, because it obeys
all the rules for this trait. Other types, such as trees or different types of
list could also implement \lstinline{Collection}. Here we try and define the
implementation of \lstinline{Collection} for \lstinline{List}:

\begin{lstlisting}[nolol]
impl<T> Collection<T> for List<T> {
    fn empty() -> List<T> {
        List::new()
    }        

    fn add(&mut self, value: T) {
        self.prepend(value);
    }
    
    fn iterate<'iter>(&'iter self) -> ListIter<'iter, T> {
        self.iter()
    }
    
    // This will not compile because there is no reference to the lifetime 
    // of the type in the generic collection. The Collection interface does
    // not provide a way of accessing the lifetime of a object associated with
    // a collection that lives for a certain length of time. 
    type Iter = ListIter<'iter, T>;
}
\end{lstlisting}

There is a large problem in this implementation. A collection contains elements
of type \lstinline{T}, and provides access to an iterator that should allow
access to those elements. In the case of \lstinline{List}, \lstinline{Iterator}
returns a reference to the element which lives for the same length of time as
the whole collection. This is problematic though because there is no way to
reference the lifetime of a collection in the trait above.

This shows how modelling a generic collections interface can be difficult. In
the case of \lstinline{Iterator}, the lifetime of the element could be passed
in to the instance of the trait. This cannot be done with the implementation of
\lstinline{Collection}. Specifically, \lstinline{List<T>} does not provide
enough information to instantiate \lstinline{ListIter<'iter, T>}.

\subsubsection{Associated Type Constructors}
To solve this problem, the type of the item that the iterator returns needs to
be associated with a lifetime. Being able to include type constructors in classes
as well as concrete types would allow this. For example:

\begin{lstlisting}[nolol]
trait Collection<T> {
    fn empty() -> Self;
    fn add(&mut self, value: T);

    // The type of Iter is now parametrized with the lifetime of the element
    // that it contains
    type Iter<'iter>: Iterator<T>;
    fn iterate<'iter>(&'iter self) -> Self::Iter<'iter>;
}
\end{lstlisting}

Now \lstinline{Iter} is a type constructor rather than a concrete type, and must
be instantiated with a concrete type before it is used. As a brief aside,
lifetime variables are actually another \textit{kind} in Rust:
\begin{figure}[H]
    \vspace{1cm}
    \textbf{Kinds} \\
    \begin{tabular}{l c p{3cm} r}
        \texttt{K} & $ ::= $ & $ * $ &      kind of concrete types \\
        & $ | $ & $ lifetime $ &            kind of lifetimes  \\
        %& $ | $ & \texttt{K} $ \rightarrow $ \texttt{K} &            kind of type operators
    \end{tabular}
    \caption{Kinds in Rust}
\end{figure}

This can be seen when passing polymorphic type arguments to functions in Rust.
For example:

\begin{lstlisting}[nolol]
fn foo<'a, T>(x: &'a T) -> u32 {
    // do stuff
}
\end{lstlisting}

This function takes a polymorphic lifetime argument and a type arguments.
\lstinline{Iter<'iter>} is a type constructor, or function at the level of
types, and hence has kind \lstinline[mathescape]{lifetime -> *}.

With associated type constructors, the collection implementation for \lstinline{List}
could look like:

\begin{lstlisting}[nolol]
impl<T> Collection<T> for List<T> {
    // same methods as before

    type Iter<'iter> = ListIter<'iter, T>;
    // Use of iter now must be parameterized by a lifetime, which can be used
    // by ListIter. Information can now be generically passed through.
}
\end{lstlisting}

This solves the problem of how to get the lifetime information to
\lstinline{ListIter}. Methods can now be written that are generic over
collections:

\begin{lstlisting}[nolol]
fn add_one<C>(xs: &C) -> C
    where C: Collection<u32>
{
    let mut xs_plus_ones = C::empty();  // Collection trait defines empty
    for &x in xs.iterate() {            // xs is a collection, which defines iterate
        xs_plus_ones.add(f + 1);
    }
    xs_plus_ones
}
\end{lstlisting}

This function could operate over trees, lists, sets, or anything else that
operates over a collection. This is very useful, however there are still
problems with this approach. In Chapter~\ref{sec:poly}, we tried to implement
\lstinline{Functor}, or an interface for a container that only provided one
method, \lstinline{map}.  The issues arise when we try and implement a generic
map function \lstinline{Collection}, similarly to how map was implemented for
\lstinline{Functor} in Chapter~\ref{sec:poly}.

\begin{lstlisting}[nolol]
fn map<A, B>(f: &Fn(A) -> B, xs: &Collection<A>) -> Collection<B> {
    let mut out = Collection<B>::empty();
    for &x in xs.iterate() {
        out.add(f(x));
    }
    out
}
\end{lstlisting}

The problem here is that this function can return \textit{any} type that
implements collection, not the same type of collection that we called
\lstinline{map} on. This means that there is very little that can be done with
the result of this function, without casting or explicit annotation. This is
especially problematic if calls to \lstinline{map} and other higher order
functions like \lstinline{filter} are chained, which is a common idiom.

Referencing Chapter~\ref{sec:poly}, what we really want to be able to do have a
signature like:

\begin{lstlisting}[nolol]
// Note that C is defined in the type parameters for this function
fn map<A, B, C: Collection>(f: &Fn(A) -> B, xs: &C<A>) -> C<B> {
    // same as before
}
\end{lstlisting}

In effect we want to say that the type parameter \lstinline{C} should itself be
a type constructor. This constructor is referenced in the rest of the signature
and the parameter \lstinline{xs} and the return type now have to be the same
type of collection. However this problem can be solved without using full-blown
higher kinded types, and instead using the associated type constructors defined
earlier in the chapter.

\subsubsection{Type Families}
Using associated type constructors and a few extra traits, the relationship
between the type constructor a list of parameters and the return type of a
function can be defined. The extra traits needed are described below:

\begin{lstlisting}[nolol]
trait CollectionFamily {
    // CollectionFamily defines one associated type constructor
    type Member<T>: Collection<T>;
}

struct ListFamily;

impl CollectionFamily for ListFamily {
    // instantiate list families with with the type of lists holding generic
    // elements
    type Member<T> = List<T>;
}
\end{lstlisting}

Family refers to the collection type without referring to its elements, e.g.\
any type of \lstinline{List} might be a collection family. This can be used to
implement a proper map function:

\begin{lstlisting}[nolol]
fn map_family<A, B, F: CollectionFamily>(xs: &F::Collection<A>) -> F::Collection<B> {
    let mut out = F::Member::empty();
    for &x in xs.iterate() {
        out.add(f(x));
    }
    out
}
\end{lstlisting}

This function can then be called with a family of types, like \lstinline{ListFamily},
as a type parameter. The function would then take a \lstinline{List<A>} as an
argument and return a \lstinline{List<B>}.

This is one of the main motivations for including higher kinded types in a
language.  However, lots of the benefits of higher kinded types could
potentially be modelled by using associated type constructors. The would solve
some problems show in Section~\ref{sec:iterators} related to lifetimes, which
would not clearly be solved by higher kinded types alone. This system
potentially has the other benefits.  Higher kinded traits, a la Haskell's type
constructor classes, would not need to be implemented. These could potentially
confuse newcomers to the language, as well as possibly making type inference
more difficult~\cite{niko}.

A future project would be to implement a language with lifetimes and type
inference, and incorporate Rust-style traits. There would be no higher kinded
types, in the style of Haskell, would not be built into the language. Traits
could only be written over concrete types. Associated type constructors could
then be added and the propositions in this section could be tested.

\section{Extensions to This Project}
Some other extensions to the lambda calculus used in this project should be
modelled in order to better reason about the influence higher kinded types
would have on Rust.

\subsection{Lifetime Elision} Lifetime elision means that lifetimes can be omitted
in cases where they are obvious from their context of use. Type signatures
often do not need to be annotated with types because default options can be
assumed that agree with what the programmer is trying to express. It is a
separate concept from inference. Elision significantly cuts down the amount a
programmer has to write and reduces unneeded information. Manual lifetime
annotations such as found in this project are not very practical for general
programming.

By default in Rust, every value in a function's parameters that has a separate
lifetime. If a function only has a single input lifetime, then that lifetime is
used for every output lifetime.

\subsection{Type Inference}
Type inference refers to automatically deducing types of expression rather than
making the programmer explicitly annotate them. It is common in statically
typed functional languages. It makes writing programs easier to write and read,
as less space used writing down types of expressions that are obvious or can
easily be inferred from context.

However, global type inference for System F has been proved to be undecidable
\cite{tapl} \cite{attapl}. In practice, languages like Haskell that do have
global type inference rely on \textit{let-polymorphism}, which has proven to be
the sweet spot between expressive power in a ease of use as a programming
language. Lots of languages with type inference, including Rust, go down this
route. If one were to design a language that included lifetime annotations and
higher kinded types, type inference would probably be a good idea as the types
of expressions and function signatures could become very large and unwieldy.
Therefore it would make sense to include let-polymorphism.

\subsection{Record Types}
Record types, called structs in Rust, are a way of combining existing types to
create more complex type. These more complex types can then be treated as a
single datatype. They are commonly used to store related data and treat that
data as a single entity.

