\chapter{Requirements}\label{sec:reqs}
The functional requirements of this project specify what this project shall do.

\section{Functional Requirements}\label{sec:functional}
The functional requirements of this project are laid out in this section.
Because the entire system is a Haskell program all of these requirements will
be implemented in Haskell.

\subsection{Parser}\label{sec:parser}
\Req{A parser for the language specified in~\ref{fig:grammar} shall be created
using the \texttt{megaparsec}\cite{megaparsec} parser combinator library.}
{Source code from a file or interactive session.}
{A data type that represents the abstract syntax of the input provided, or an
error message pointing to the location of any syntax errors.}
{An error message with a line and column number and a message.}

\subsection{Simple Type Checker}\label{sec:typecheck}
\Req{A simple type checker that ensures that simple mistakes are not make,
e.g.\ using a function type where a numerical type is expected.} {A
syntactically valid (according to Figure~\ref{fig:grammar}) AST of a program as
a Haskell data type.} {The same AST of the program that confirms the typing
rules of the language.} {An error with a message that provides some indication
of what went wrong.}

\subsection{Ownership Checker}\label{sec:owncheck}
\Req{An ownership checker ensures there exactly one binding to a resource all
the time.} {A syntactically valid (according to figure~\ref{fig:grammar}) AST
of a program as a Haskell data type.} {The same AST of the program that
confirms the ownership rules of the language.} {An error with a message that
provides some indication of what went wrong.}

\subsection{Borrow and Lifetime Checker}\label{sec:borrowcheck}
\Req{A checker that ensures that references that borrow ownership from another
type last longer than the resource they borrow, and that there are only mutable
references OR exactly one mutable reference at any one point.} {A syntactically
valid (according to figure~\ref{fig:grammar}) AST of a program as a Haskell
data type.} {The same AST of the program that confirms the borrowing rules of
the language.} {An error with a message that provides some indication of what
went wrong.}

\subsection{Kind Checker}\label{sec:kindcheck}
\Req{Ensures that all type constructors using in a program have the correct
number and kind of arguments.} {A syntactically valid (according to
figure~\ref{fig:grammar}) AST of a program as a Haskell data type.} {The same
AST of the program that confirms the kinding rules of the language.} {An error
with a message that provides some indication of what went wrong.}

\subsection{Evaluator}\label{sec:evaluator}
\Req{A call-by-value evaluator of the language that will reduce a syntactically
valid expression.} {A syntactically valid and type-checked AST of a program
represented as a Haskell data type.} {The final resulting value of evaluating
the AST.} {A description of any runtime errors that occur within the program.}

\subsection{Interactive Interpreter}\label{sec:interpreter}
\Req{An interactive interpreter that type checks and then evaluates entered
expressions.} {Source code as entered by the user.} {The resulting value of
evaluating the entered expression, some error.} {An parsing, type, or runtime
error message.}

\subsection{Load a file}\label{sec:load}
\Req{Provided with a path, the program loads a text file containing source
code.} {A path provided by the user.} {The source code as a string.} {An error
reporting a file not found or any other errors.}

\Note{Should this section be related to KEY TESTS? Maybe this section should
exist but should reference that chapter}
\section{Acceptance Criteria and Testing}
The acceptance criteria of this program correspond to the functional
requirements in Section~\ref{sec:functional}. The finished project should pass
the tests laid out in this section.

\subsection{Parsing}
\Accept{\ref{sec:parser}} {The program should be able to parse valid source
code and correctly report any errors that are encountered.} {Test numbers}

\subsection{Type checking}
\Accept{\ref{sec:typecheck},~\ref{sec:owncheck},~\ref{sec:borrowcheck},~\ref{sec:kindcheck}}
{The type checker should detect any errors in the program.} {Test numbers}

\subsection{Evaluating}
\Accept{\ref{sec:evaluator},~\ref{sec:interpreter},~\ref{sec:load}} {Source
code, provided by a file or through the interactive interpreter, should be type
checked and evaluated.} {Test numbers}

