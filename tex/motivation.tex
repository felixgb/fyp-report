\chapter{Motivation}\label{sec:motivation}
CHAPTER INTRO HERE

\section{Higher Kinded Types}
This chapter introduces Higher Kinded Types, also known as higher order
polymorphism, and show what kinds of problems a language that includes these
features can solve.

\subsection{Parametric Polymorphism}\label{sec:generics}
First order parametric polymorphism, known as Generics in Java, are types that
are parametrized over some other type. This kind of polymorphism allows for the
definition and use of functions that behave uniformly over all types.  They can
be used to write code that can be checked for safety at compile time. For
example, without generics, a list could be used like:

\begin{lstlisting}[caption=Runtime error that could be avoided, language=Java, label={lst:javaerr}]
List l = new ArrayList();
l.add("This is a string");
Integer i = (Integer) l.get(0); // Run time error here
\end{lstlisting}

This is problematic because a list in Java can contain any type of object, but
methods that the \lstinline{List} object provide must know about the type of
object that the list contains. If those methods do not have a way of knowing
what kind of object a list contains then calling the methods will not be type
safe, as demonstrated in Listing~\ref{lst:javaerr}. The same code written with
Java's Generics will produce a compile time error:

\begin{lstlisting}[caption=Compile time error, language=Java, label={lst:generr}]
List<String> l = new ArrayList<String>(); // Now the list has been parametrized with a type
l.add("This is a string");
Integer i = l.get(0);                     // Compile time error here.
\end{lstlisting}

Compile time errors are much more desirable then runtime errors because they
can be caught and fixed predictably, unlike runtime errors which may happen at
unpredictable times. The code in Listing~\ref{lst:generr} parametrizes the
\lstinline{List} type with the \lstinline{String} type, and hence the last line
produces a compile time error as the list's \lstinline{get} method returns a
\lstinline{String}. 

Parametric polymorphism is an important addition to statically typed programming
languages. However, Generics in Java and in lots of other programming languages
have the limitation that types can only be parametrized with other types. 
This leads to some important limitations.

\Note{Talk about formalization, System F}
\subsection{Higher Order Polymorphism}
As mentioned in Section~\ref{sec:generics}, first order parametric polymorphism
can be very useful in expressing certain concepts succinctly in programming 
languages. However, there are limitations. This section will attempt to 
demonstrate one shortfall of first order polymorphism and then show how the
problems can be solved with higher order polymorphism.

\subsubsection{Functors}
A Functor is important concept in modern programming languages. Most languages
have constructs that can be thought of as Functors, such as lists, optional types,
trees, and other constructs that can be mapped over. Anything that acts as a
container for another type and provides a function for mapping a function over
that contained type is a Functor. They are an important concept because they
provide a common interface to working with any type that acts as a box for
another type. This means that programs can be written more generically and is
more open to changes and modifications.

Some languages seek to reify the general concept of functors so that aspects of
a program can be checked for correctness at compile time. However, expressing
Functors in a generic way requires a more expressive type system, specifically
a type system that incorporates higher order polymorphism. Here is an example
of an attempt to define a general Functor interface in Java:

\begin{lstlisting}[caption=An attempt to define Functor in Java, language=Java, label={lst:javaFunctor}]
interface Functor<A> {
    Functor<B> map(Function<A, B> f);
}
\end{lstlisting}

The code in Listing~\ref{lst:javaFunctor} has one main issue. The
\lstinline{map} method defined here may return any type that implements
\lstinline{Functor}, not the necessarily the same class as the one that the
method has been called from. This means that there is no type-safe way of
calling any method on the result of calling the \lstinline{map} function.

In order to define a generic Functor interface, a way of referencing the type
constructor that is being used as a functor is needed. In other worlds, the
programming language needs type constructors that can be parametrized with
other type constructors. Generics in Java provide a way of making types like
\lstinline{Integer}, \lstinline{String}, and even \lstinline{List<Integer>}
first class, but type constructors like \lstinline{List} must be applied to
some concrete type before they can be abstracted over.

\begin{lstlisting}[caption=Functor as defined in Haskell., language=Haskell, label={lst:haskellFunctor}]
class Functor f where
    fmap        :: (a -> b) -> f a -> f b
\end{lstlisting}

Listing~\ref{lst:haskellFunctor} shows how Functor can be defined in a Haskell,
a language that does allow higher order polymorphism. Here, \lstinline{f} is a
variable that references a type constructor. This class in Haskell specifies
that anything that instanciates it must provide a single function,
\lstinline{fmap}.  \lstinline{fmap} takes a function as a parameter that takes
a value of type, \lstinline{a}, and returns a value of type \lstinline{b}. It
then takes another argument of type \lstinline{f a}, or a type that the type
constructor \lstinline{f} has been applied to. For example, it would take a
value of type \lstinline{Maybe Int}, \lstinline{Maybe} being the type
constructor \lstinline{f}. It would then return a value of type \lstinline{b},
wrapped in the same type constructor \lstinline{Maybe}.

The ability to use type constructors as first class in a programming language
allows programs to be written in a more succinct, patterned, and generic manner.
Examples like Functor show how a generic interface to work with constructs
commonly found in software engineering can be achieved with the use of Higher
Order Polymorphism.

\subsubsection{Kinding}

Higher order polymorphism can be thought of as the polymorphic lambda calculus
`one level up', and as such the language at the level of types also needs to 
have a type system in order to prevent wrong expressions from being created.

A system that allows for abstracting over type constructors needs some way
of enforcing type constructors are not applied to types in a way that does not
make sense, e.g.\ \lstinline{Integer Integer} does not make sense because the
integer types does not take any type arguments in order to become a concrete
type that can be inhabited by values. Alternately just \lstinline{List} cannot
be instantiated with a value because it need one more type constructor, 
the type that will be contained within that list.

What is needed is essentially a type system for types. This is the notion of
the kind of a type. Types can have the kind of $ * $, in which case
they are concrete and can be inhabited by values, or they can have kind
$ * \rightarrow * $, or a function from types to other types. Types like
\lstinline{Integer} or \lstinline{List Integer} have kind $ * $ however
type constructors like \lstinline{List} have kind 
$ * \rightarrow * $ because they need to take one more concrete
types before they can be used as a value. This system of kinding enforces that
all types are well formed.

\begin{figure}[H]\label{fig:kinds}
    \centering
    \begin{tabular}{l c p{3cm} r}
        $ K $ & $ ::= $ & * &                   Concrete types \\
      & $ | $ & $ K \rightarrow K $ &           Type functions from types to types \\
    \end{tabular}
    \caption{Syntax of types}
\end{figure}

%%% unfuck this thing

\begin{figure}[h]\label{fig:kindexamples}
    \centering
    \begin{tikzpicture}[every node/.style={draw=none}]
        \node at (0, 0) (k){$ * $};
        \node at (13, 0) (k2k){$ * \rightarrow * $};
        \node at (17, 0) (k2k2k){$ * \rightarrow * \rightarrow * $};
        \node at (24, 0) (kk2k){$ (* \rightarrow *) \rightarrow * $};

        \node[below left=of k] (int) {\texttt{Int}};
        \node[below right=of int, xshift=-1cm] (listInt) {\texttt{List[Int]}};
        \node[above right=of listInt, xshift=-2cm] (pairInt) {\texttt{Pair[Int, Int]}};

        \node[below=of int, yshift=-2cm] (val) {\texttt{1}};
        \node[below right=of val, xshift=-2cm] (listVal) {\texttt{[1, 2, 3]}};
        \node[above right=of listVal, xshift=-2cm] (pairVal) {\texttt{(1, 2)}};

        \node (proper) [draw, fit= (k) (int) (listInt) (pairInt) (val) (listVal) (pairVal)] {};
        \node[yshift=3.0ex] at (proper.north) {\textit{Proper}};

        \draw[->] (val) -- (int);
        \draw[->] (listVal) -- (listInt);
        \draw[->] (pairVal) -- (pairInt);

        \draw[->] (int) -- (k);
        \draw[->] (listInt) -- (k);
        \draw[->] (pairInt) -- (k);

        %%% first order
        \node[below=of k2k] (list) {\texttt{List}};
        \node[below=of k2k2k] (pair) {\texttt{Pair}};

        \draw[->] (list) -- (k2k);
        \draw[->] (pair) -- (k2k2k);

        \node (first) [draw, fit= (k2k) (k2k2k) (list) (pair)] {};
        \node[yshift=3.0ex] at (first.north) {\textit{First Order}};

        %%% higher order
        \node[below=of kk2k] (functor) {\texttt{Functor}};

        \draw[->] (functor) -- (kk2k);

        \node (higher) [draw, fit= (functor) (kk2k)] {};
        \node[yshift=3.0ex] at (higher.north) {\textit{Higher Order}};

        \node[left=of k, xshift=-3cm] (kinds) {\textit{Kind}};
        \node[left=of int] (type) {\textit{Type}};
        \node[left=of val] (value) {\textit{Value}};
    \end{tikzpicture}
    \caption{Some types and their associated kinds.}
\end{figure}

\section{Problems with Existing Memory Management Techniques}
Most programming languages allow for dynamic memory management where the
program can request portions of memory at run time and free that memory when it
is no longer required. This operation can be performed manually as in the case
of languages such as C or automatically in languages with a garbage collector
such as Java.

\subsection{Manual Memory Management: \lstinline{malloc} and \lstinline{free}}
Some languages rely on manual instructions inserted by the programmer in order
to allocate and deallocate regions of memory. Most implementations of the C
programming language provide a group of library functions for this purpose,
which include \lstinline{malloc} and \lstinline{free}. As mentioned, these
constructs are insert manually by the programmer. This can lead to the issues
described below.

\subsubsection{Dangling References}\label{sec:dangle}
Deallocating storage may lead to dangling references, where some reference 
exists that points to memory that has been deallocated. This is nearly always
unintentional, however it is an easy for a programmer to make this mistake when
writing a program. Using the value of deallocated storage can lead to nefarious
bugs and most languages consider using deallocated storage to
be undefined behaviour.

\lstinputlisting[
    caption={A C program that leaves \texttt{p} pointing to deallocated 
    memory.},
    label={lst:cdangle}
]{codeexamples/dangle.c}

In the Rust programming language, references to values must live longer than
the resource that they refer to~\cite{rust-borrowing}. The same program as the 
C dangling pointer example (Listing~\ref{lst:cdangle}) written in Rust does not
compile as it does not meet this restriction.

\lstinputlisting[
    caption={A Rust program that does not type check.}
]{codeexamples/dangle.rs}

\subsubsection{Memory Leaks}\label{sec:leak}
Memory leaks occur when heap storage that is no longer usable is not
deallocated. This can lead to software allocating more memory for itself and in
the worst case can lead to a program consuming all available memory.

\lstinputlisting[
    caption={A C program which leaks memory.}
]{codeexamples/leak.c}

\subsection{Automatic Memory Management: Garbage Collection}
Techniques have been invented in order to mitigate errors related to manual
memory management. Garbage collection is one such method.  Garbage is a term
for storage that has been allocated on the stack but is no longer reachable by
the program, e.g.\ the storage is pointed to by a variable that has gone out of
scope. Automatic garbage collection alleviates the programmer of explicitly
managing memory. However there are several downsides to garbage collection. For
example, the garbage collector will be invoked to collect unusable memory
outside of the control of the programmer. When it is running it affects the
execution of the program for an indeterminate amount of time. This hang is not
acceptable in real--time systems.

\section{A Solution in the Type System}
The Rust programming language achieves memory safety through its ownership
rules and borrowing system~\cite{rust-borrowing}. The ownership system allows
Rust to achieve this memory safety by enforcing typing system rules at compile
time. This means that there is no run time overhead in a Rust program as there
would otherwise be in a garbage collected language.

\subsection{Ownership, Borrowing, and Lifetimes} \label{sec:regions}
In Rust, a resource must have one owner, a stack variable. When a variable goes
out of scope, the resource that it owns is automatically freed. There cannot
be more than one variable pointing to a given resource at any time. The reason
for the ``use of a moved value'' error often encountered in code written
by novice Rust users is because of this rule. This prevents any two parts
of a Rust program from accessing a section of heap memory at the same time.

Rather than passing ownership of a resource around between functions, Rust
allows for a reference to be borrowed by another scope temporarily. When
references go out of scope the resource that they point to (have borrowed) do
not get automatically freed. This means that after a function with a
reference passed in as as parameter returns the resource can be used again.
The scope of borrow must always be shorter than that of the resource which it
borrows. This concept is formalized in Rust as Lifetimes. The type system will
enforce that a borrow is not made of a resource that will be freed before the
borrow ends.

There are two kinds of references: immutable and mutable. There many be many
immutable borrows of a resource at once or a single mutable borrow. This also
prevents heap data from being accessed at the same time at different points in
the program.

This system where resources are freed after the single owner goes out of scope,
as well as borrows always having shorter lifetimes than original resource
mean that Rust can enforce the memory safety of programs. These concepts are
built into the type system.
