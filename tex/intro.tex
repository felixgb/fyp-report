\chapter{Introduction}\label{sec:intro}

\section{Project Aims}
The purpose of this project is to design a type system for a programming
language that incorporates Higher Kinded Types with a system that guarantees
memory safety at compile time. Higher Kinded types are necessary to neatly
express certain desirable programming features. Rust is a programming language
that provides memory safety through compile-time checks. This aspect of Rust is
known as the \textit{ownership} model, of which \textit{lifetimes} and
\textit{borrowing} are key concepts \cite{rust}. This project attempts to
implement a type checker for this language as a Haskell program, and
investigate how higher kinded types and the Rust ownership model interact.

\section{Safety in Programming Languages}
As programs become larger and more complex, methods have been devised to detect
and mitigate classes of errors. Ideally, all errors would be detected before a
program is run. To this end, type system have been developed that either
eliminate certain classes of errors or make certain styles of bug-prone
programming hard or impossible. While doing this, it is important that a type
system still benefits the programmer by making the language more expressive.

\subsection{Simple Types}
Type systems are designed to help programmers reason about behaviours in the
programs that they write. Types restrict the kinds of statements that may be
expressed, the most obvious reason for doing this is to detect certain types of
errors before a program is executed. Small, obvious errors, found in
expressions such as \lstinline{"this sentence" * 69} are easily detected,
because multiplication makes no sense when applied to a strings of characters
and a number.  However, modern type systems are capable of doing far more. 

\subsection{Polymorphism}
As programs become larger and more complex, separating parts of programs that
do not need any knowledge of each other becomes more important. It is much
easier to maintain a program composed of separate modules that have no
knowledge of the inner workings of one another. Modules may interact through
well defined interfaces and in this way may be swapped out or replaced much
more easily. One may re-write code that conforms to an existing interface
without knowing about another module's implementation details. A type system can
be used to define this interface and to ensure that new code behaves as
expected. Polymorphism, discussed in detail in Chapter~\ref{sec:poly}, is
one way of allowing more flexibility in these interfaces between parts of a
program while retaining correctness.

\subsection{Memory Safety}
Some programming language features may introduce whole new families of errors.
Allowing a programmer to manually allocate and deallocate space for storing
objects in memory is known to be a common source of bugs. Languages like C have
a reputation for being difficult to write bug free programs in because of
manual memory management. Memory leaks may occur if a programmer requests some
storage in memory but does not deallocate it. This can lead to programs
consuming unreasonably large amounts of memory. Referencing some location in
memory after it has been deallocated (and possibly allocated with some new
object) is another common pitfall associated with manual memory management.
Garbage collection, or the automatic allocation and deallocation of objects in
heap memory, may alleviate some of these issues.  However, it can incur a
sometimes unacceptable runtime overhead especially in software systems where
responsiveness is key.

Types can also play a role in preventing these kinds of errors. Restricting
where a programmer can request or reference some location in memory allows a
program to be statically analysed before it is run, and any sections of code
that could cause memory related issues are shown as compile time errors.  A
type system can form the basis of this restriction, as shown in
Section~\ref{sec:regions}

\section{The Lambda Calculus}
In studying these type systems it is often useful formalize them in a system
with well known and well understood behaviour. In this case the system is the
lambda calculus, which functions as a miniature programming language and as a
formal model for understanding the behaviour of types. The lambda calculus and
how it pertains to this project is described in Section~\ref{sec:lambdacalc}.
Higher Kinded Types, or higher order polymorphism, have been well described in
System F $\omega$, itself a typed lambda calculus~\cite{tapl}.

\section{Goals of This Project}
A parser, type checker, and simple interpreter will all be implemented along
with a test programs demonstrating some intended behaviours. Haskell will be
used as the language of implementation, and is described in
Chapter~\ref{sec:impl} along with the reasons for choosing it.  The design of
the language is described in Section~\ref{sec:formal}.  Some of the main goals
of this project are:

\begin{itemize}
    \item
        Developing a very simple programming language that incorporates dynamic
        memory allocation, modelling functions that operate on a heap. 
    \item
        A system restricting the allocation, use, and deallocation of memory,
        encoded into the type system. This aspect is based on the type system
        of the Rust programming language, which guarantees memory safety
        through associating memory tied to function scope.
    \item 
        A type system that allows for higher order polymorphism, described in
        Chapter~\ref{sec:poly}.  Formalised in System F $\omega$, it allows
        for a neat implementation of very generic, high level programming
        concepts such as Monads and Functors.
\end{itemize}

The notion of an addressable area of memory, as well as mechanisms for
allocating and deallocating areas of that memory, will be added to the base
lambda calculus.  The implementation of the type system, as well as the parser,
build system, test framework, and interpreter are described in
Chapter~\ref{sec:impl}. A formal description of the language can be found in
Chapter~\ref{sec:formal}.

\section{Project Overview}
The rest of this report is divided in several chapters.
Chapter~\ref{sec:ethics} lists the ethical considerations of this project.
Chapter~\ref{sec:poly} introduces polymorphism before introducing Higher Kinded
Types. Memory allocation and the problems associated with it is introduced in
Chapter~\ref{sec:memory}.  Chapter~\ref{sec:formal} then introduces the lambda
calculus as way of reasoning about type systems. The design of the language is
introduced by building up features from existing languages and then
incorporating them into one final grammar.  Chapter~\ref{sec:requirements}
formalizes what should be accomplished by the type checker to be implemented.
Software engineering issues are tackled in Chapter~\ref{sec:impl}, including
the test framework and build system used. Key tests demonstrating desired
behaviour are given in Chapter~\ref{sec:testing}. Issues about the
effectiveness of this type system are described in
Chapter~\ref{sec:evaluation}. Finally, unmet requirements and further work to
be done is covered in Chapter~\ref{sec:conclusion}.
