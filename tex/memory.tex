\chapter{Memory Management}\label{sec:memory} 
Part of this project is designing a compile time system for memory management.
Before describing how this system should work, an overview of existing memory
management techniques is given. Most programming languages have the ability to
dynamically allocate memory on the \textit{Heap}, which is a very useful
language feature.  However, the allocation, dereferencing, and freeing of heap
memory is problematic and this chapter will describe some of the issues with
heap allocated memory.

\section{Memory Usage and Computational Effects}
Most practical programming languages allow for storing and referencing areas
in memory. When an expression is evaluated, it may have an effect on memory as
well as returning a result. For example, an assignment statement in a programming
does not have any obvious result, but instead operates on values in an area of
memory that is accessible to other parts of the program. A language that allows
for these sorts of computational effects is referred to as \textit{impure} and
one of the goals for this project is to make these features safer to use.

\subsection{Stack and Heap Memory}
There are usually two kinds of memory that may be used to store values: stack
memory and heap memory. Stack memory is commonly used to store values for
function calls~\cite{appel}. A function is called with parameters referred to by variables,
which are in scope for the duration of the function call. These local variables
hold the result of evaluating some expression, which is typically stored in
stack memory. This has the advantage that memory is very predictably allocated
and deallocated, and this operation of allocating and deallocating can be
accomplished very quickly. 

However, dynamic memory allocation is sometimes needed when multiple functions
need to refer to or mutate the same value; such as is the case with global
variables. Typically dynamic memory allocation refers to memory management on 
the heap.

These values live longer than any individual scope and therefore
it does not make sense to store them on the stack. Generally, heap memory is
used when some value needs to be alive for a longer than a single function's
execution. However this memory cannot be predictably allocated and deallocated.
Programming languages have several strategies for managing heap memory.


\subsubsection{Basic Operations}\label{sec:operations}
The basic operations needed in a system with memory allocation are:
\begin{description}[style=nextline]
    \item[Allocation]
        Allocation refers to creating an area in memory to store the result of
        an expression. This is commonly seen is the form of variable
        assignment, for example \lstinline{x := ref 2 + 3;}. Here the
        expression \lstinline{2 + 3} is evaluated~\footnote{In languages with
        \textit{strict} evaluation, the expression is evaluated before the
        result is bound to the variable. This is not the case in all
        programming languages.} and the result is stored in the area of memory
        referred to by the variable \lstinline{x}. The value referred to by
        \lstinline{x} can then be used at some later point in the program.
        Variables may also be reassigned, for example \lstinline{y := x}.
        \lstinline{y} would then point to the same location that \lstinline{x}
        points to.
    \item[Dereference]
        Once an area of memory has been allocated, and is pointed to by a
        variable, some way of using the expression stored at that location is
        needed. Dereferencing refers to getting the expression stored at a
        location in memory. If we wanted to use \lstinline{x} at some point
        after the expression \lstinline{x := ref 2 + 3} had been evaluated, we
        could write \lstinline{!x}. The result of that expression would be 5,
        because \lstinline{2 + 3} has been evaluated and the result stored in
        the area of memory that is pointed to by \lstinline{x}. The dereference
        operator, \lstinline{!E}, means ``go to the location in memory,
        \lstinline{E}, and get the value stored there''. 
    \item[Deallocation]
        Deallocation refers to freeing memory once it no longer needed. It is
        common to not want to use memory for the whole lifetime of a program.
        If memory was never freed programs would use far more memory than they
        needed, which would not be practical. In order to solve this problem
        memory may be deallocated once a variable goes out of scope, which is
        common for stack memory, or we could introduce an explicit deallocate
        operator, called \lstinline{free}. This operator would take a memory
        location and mark that location as unused so that a future allocation
        could reuse the same space.
\end{description}

\section{Problems with Dynamic Memory Allocation}\label{sec:problems}
Although dynamic memory allocation is a very useful language feature it is also
a notorious source of bugs. Compared to stack memory allocation, there is no
enforced regular pattern of allocation and deallocation. Either the programmer
must explicitly state when to allocated and deallocate memory, or an automatic
system is used to management memory, called a garbage collector. If heap memory
is manually managed, such as is the case in languages like C, there are some
common bugs, which are described in this section.

\subsection{Dangling References}\label{sec:dangle}
Deallocating storage may lead to dangling references, where some reference 
exists that points to memory that has been deallocated. This is nearly always
unintentional, however it is an easy for a programmer to make this mistake when
writing a program. Using the value of deallocated storage can lead to nefarious
bugs and most languages consider using deallocated storage to
be undefined behaviour.

\lstinputlisting[
    caption={A C program that leaves \texttt{p} pointing to deallocated 
    memory.},
    language=C,
    label={lst:cdangle}
]{codeexamples/dangle.c}

In this example the function \lstinline{dangle()} returns the address of variable
that is deallocated when the function terminates. This means that part of the
program that attempts to use the value returned by \lstinline{dangle()} will
result in result in undefined behaviour.

\subsection{Memory Leaks}\label{sec:leak}
Memory leaks occur when heap storage that is no longer usable is not
deallocated. This can lead to software allocating more memory for itself and in
the worst case can lead to a program consuming all available memory.

\lstinputlisting[
    caption={A C program which leaks memory.},
    language=C
]{codeexamples/leak.c}

In this C program memory is allocated in the function \lstinline{allocate()}.
The C library function \lstinline{malloc(size_t size)} allocates \lstinline{size}
bytes on the heap and returns a pointer that area. However, in this example,
the pointer that is returned, bound to the variable \lstinline{a}, is not used
and never returned. Therefore the memory has been allocated on the heap but
there is no way of referencing it once the function \lstinline{allocate()} goes
out of scope. If the function were used in a loop, heap memory would very quickly
be used up pointlessly! 

\subsection{Double Free}\label{sec:doublefree}
Once a programmer no longer needs some dynamically allocated memory they should
deallocate it so the memory space is not wasted. This is can be accomplished by
calling a provided library function, such as C's \lstinline{free(void *ptr)}.
This function takes an address of some space in heap memory as a parameter and
makes it allocatable again. However, if the same address in memory were to be
freed twice, then some other object that may now reside in that space may be
prematurely destroyed. For example:

\lstinputlisting[
    caption={Mistakenly freeing the same memory more than once.},
    language=C
]{codeexamples/doublefree.c}

In this example space is allocated which is pointed to by \lstinline{p}. Once
this storage is free, space is allocated for \lstinline{q}. The space allocated
for \lstinline{q} might be the same space that was originally allocated for
\lstinline{p}.  When \lstinline{p} is freed again, \lstinline{q} might be
accidentally destroyed.  Any future attempt to use \lstinline{q} again after
this would result in undefined behaviour.

\section{Potential Solutions}
Several design patterns and technologies have been developed for tackling the
dynamic memory management issues described in Section~\ref{sec:problems}.  This
section attempts to describe some of these techniques, while noting why they
may not an ideal solution.

\subsection{Garbage Collection}
Garbage refers to heap allocated storage that is no longer reachable by the
programmer. This memory can no longer be used and should be deallocated. It is
important to note that garbage collection is not performed by the compiler, but
by a runtime system~\cite{appel}. Languages with garbage collection do not have
manual deallocation of memory, but instead rely on a system that attempts to
track when objects are no longer reachable.  The programmer therefore does not
have to manually annotate the program and tell the compiler when to release
memory, i.e.\ constructs like C's \lstinline{free} are not needed. Garbage
collection effectively solves problems related dangling pointers, double free
bugs, and certain kinds of memory leak.

Many modern language rely on garbage collection for memory management. Because
of improvements in safety and ease of use Java, Python, Scala, Haskell, and
many others all rely on garbage collection. This has a notable effect on
programmer efficiency~\cite{garbage}. However, use of a garbage collector does
not come without cost. Some issues are:

\begin{itemize}
    \item
        A programmer can have some insight as to where objects should be freed
        in a program and may know how long an object should exist for. The
        garbage collector may not be able to make this insight. As a result a
        program written in a garbage collected language can use far more memory
        than than a program in a language with manual memory management written
        by an expert.
    \item
        Significant computational overhead is needed in a program because
        garbage collectors must always account for the worst case scenario.
        Because they need to be running at the same time as the program that
        they manage they can significantly decrease performance~\cite{garbage}.
        In lots of applications the speed of execution does not matter, but for
        certain types of applications the detriment to performance is
        unacceptable.
    \item
        Program execution may be interrupted in order to collected unused
        memory. The programmer cannot predict when these pauses may happen.
        Certain types of real-time or interactive applications should not be
        interrupted, so cannot be programmed in garbage collected languages.
\end{itemize}

Because of these issues, use of a garbage collected language is not always the
correct decision.

\subsection{Resource Acquisition is Initialization}\label{sec:raii}
Resource Acquisition Is Initialization, or RAII, is a technique, not a
technology, which connects the lifetime of a resource, such as heap memory, to
the lifetime of an object~\cite{cppref}. It is a way of managing dynamically
allocated memory in languages with object oriented features.

\subsubsection{Lifetimes}
The lifetime of an object or resource refers to the length of time from when an
object is initialized to when it is deallocated. It is used in object oriented
programming to refer to the length of time that between an object's creation
and destruction. 

Lifetimes can be used for resource management, which is the essence of of the
RAII idiom. Lifetimes refer to the duration that an object is usable for.
Resources are allocated when an object is created, and deallocated when an
object is destructed. An object's lifetime may be tied to its enclosing scope,
and in this way resources may be freed when an object goes out of scope.
Resources will be freed even the case of exceptions causing a function to exit
early. In this manner resources will always be deallocated when an object goes
out of scope, effectively solving dangling issues related to dangling
pointers~\cite{rust}. However, it is still a problem if these ideas are not
enforced by the programming language as part of the type system. Programmers
may forget to implement these ideas or may not be aware of them. However, some
of these ideas can be statically checked by a type checker.

\section{Enforcing Memory Safety in the Type System}
The idea that all resources have an owner is key. This is the core concept
from~\ref{sec:raii}: if a resource's existence is tied to an scoped variable
then the resource can be deallocated when that variable goes out of scope.  If
any given resource is associated with a \textit{single} variable then there is
no way of loosing track of it. In order for this system to work them no memory
must be aliasable, which is should not be because all resources are tied to a
single owner.

\subsection{Ownership and Affine Type Systems}
Affine types are a development of Linear type systems. Linear type systems
ensure that every variable is used exactly once. \cite{ownership} In an affine type system, a
variable may be used at most once, i.e.\ zero or one times. This has
applications for managing resources in a program that do not persist the whole
time that the program is running, e.g.\ file handles or heap allocated memory

This relates to the concept of ownership. How?
\cite{attapl} \cite{tovAffine}.

lifetime notes: \\
the duration of a thing's usefulness \\
every reference has a lifetime as part of its type \\
references cannot be used outside of this lifetime \\
the time in which some thing is executing \\
use lifetimes to think about when things are occurring simultaneously \\
lifetimes also solve returning pointer into stack to caller function,
borrow of a resource cannot live past the length of the original resource

Becuase resources allocation and deallocation is tied to function scope, ... \\

danger from 1. aliasing and 2. mutation \\

aliasing hide dependencies \\

three things: ownership, shared imm references, and mut references \\

if a resource has no owner, you can free automatically. \\
memory is freed when reference count drops to zero \\

borrowing: you don't want to give ownership, but someone else can use it for \\
while

all memory has a clear owner \\

always have a clear owner (single owner) so you know when to free it \\

Useful because there is no runtime \\

\section{Introducing Rust}

\subsection{Ownership}\label{sec:regions}
Values in Rust are bound to a single \textit{owner}, which is the variable to
which it is bound. When the owning variable goes out of scope, the value is
freed. For example:
\begin{lstlisting}[nolol, label={lst:rustScope}]
{
    let x = 69;
}
\end{lstlisting}
When \lstinline{x} goes out of scope the value associated with it will also be
released from memory. Because there can only be a single owner associated with
a value, aliasing a variable is not allowed. This prevents any two parts
of a Rust program from accessing a section of heap memory at the same time.
\begin{lstlisting}[nolol, label={lst:rustScope}]
{
    let x = Thing::new();
    let y = x;
    println("()", x);
}
\end{lstlisting}
The above code does not compile because \lstinline{x} has been moved to
\lstinline{y}, and there is an attempt to use \lstinline{x} in the print
statement. Using a variable as a function argument will also make it not
available for use again. If a function uses some value that has been passing in
as a argument it must return that same value if the scope outside that
function uses it.
\begin{lstlisting}[nolol, label={lst:rustScope}]
// value is comes from outside the scope of this function
fn foo(value: Bar) -> Bar {
    // do stuff with value, returning it and hence handing back ownership
    value
}
\end{lstlisting}
This is not practical, and gets even less practical once there are more
arguments that must be passed back to the caller of the function. To this end
Rust allows references to a value to be created. Creating a reference to a value
in Rust is called borrowing. A reference to a value may be created and passed
in to a function that uses it in some way, and that function does not need to
thread the ownership of that variable back to the caller.
\begin{lstlisting}[nolol, label={lst:rustScope}]
// instead of passing value1 and value2 in by value, create references to values
fn foo(value1: &Bar, value2: &Bar) -> u32 {
    // do stuff with value1 and value2 and return the answer
    123
}
\end{lstlisting}
There are two kinds of borrows: immutable and mutable. There may be many
immutable borrows of a resource at once or a single mutable borrow. This also
prevents heap data from being accessed at the same time at different points in
the program. The scope of borrow must always be shorter than that of the
resource which it borrows. 

This system where resources are freed after the single owner goes out of scope,
as well as borrows always having shorter lifetimes than original resource
mean that Rust can enforce the memory safety of programs. These concepts are
built into the type system.
