\chapter{Implementation}\label{sec:impl}
This chapter describes how the ideas and type system described in
Chapter~\ref{sec:formal} are implemented. As stated in
Chapter~\ref{sec:requirements}, one of the goals for this project is to build a
program that will be able to parse, type check, and evaluate a language based
on the lambda calculus. The most involved phase of implementation and the focus
of this project involves the type checking phase, which includes lifetime
checking of pointers and references, polymorphic type checking, and checking
that the language of types themselves is well formed, or kind checking.

The requirements of this project influence the choice of the language of
implementation. Haskell was chosen as it is well suited to the tasks of
parsing, abstract syntax tree declaration, and is in general a practical
language to work with.

\section{The Haskell Language}
Haskell is a general purpose, statically typed, functional
language~\cite{haskell}. It has several desirable features for a language
implementation language. These include:

\begin{itemize}
    \item Algebraic data types, which are very good at representing abstract syntax
    trees, used to describe programming languages.
    \item Pattern matching, which is useful for deconstructing abstract syntax trees.
    \item A wealth of effective parsing libraries, convening several paradigms.
    \item Idiomatic monadic programming, which can be used to reduce error handling
    and state threading boilerplate out of the logic of the program.
\end{itemize}

It also has the benefit of having higher kinded types built into the language,
making it ideal for testing potential features of the implemented language.
The language described in Chapter~\ref{sec:formal} has been implemented as a
Haskell program. The rest of this chapter describes how this was accomplished
in more specific detail.

\section{Stack, Haskell's Build System}
Stack was used as the build tool for this project~\cite{stack}. This made it
easy to compile and test the project as well as managing libraries. Several
commands were very useful in managing the project, including \lstinline{stack build}
to compile all changes to the project and \lstinline{stack test} which
ran all of the tests for the project. The lexer and parser generator source
files were not Haskell source files but instead specify how Haskell source
files should be built. These tools are described in detail in later sections
but Stack was used to automate the building of these tools as well. Finally,
Stack provides an interface to an interactive interpreter for Haskell, GHCI.
This tool was used extensively to test parts of the project as they were being
developed.

\section{Parsing}
A lexer generator, Alex~\cite{alex}, and parser generator, Happy~\cite{happy},
were used in this project. Using these tools in combination made parsing the
language into the Haskell representation of the abstract syntax simple.  The
grammars provided for the parser generator are a very close approximation of
Haskell data type. Both Alex and Happy generate Haskell files when run which
implement the specified grammars.


\section{Componests of the Type Checker}
\subsection{Terms}
The Haskell representation of terms of the language are given in
Listing~\ref{lst:terms}.  Type lambdas and Lifetime lambdas can be seen as
first class citizens here.

\begin{lstlisting}[caption=Haskell representation of Terms., language=Haskell, label={lst:terms}]
data Term
    = Var String
    | Lit Int
    | Lam String (Lifetime, Type) Term
    | App Term Term
    | TyLam String Kind Term 
    | TyApp Term Type
    | LiLam String Term
    | LiApp Term Term
    | Lt Lifetime
    deriving (Eq, Show)
\end{lstlisting}

\subsection{Kind Checker}
The kind checker makes sure that the type expressions in a supplied program are
well--formed. The Haskell abstract data type that represents kind expression is
show in Listing~\ref{lst:kinds}.

\begin{lstlisting}[caption=Haskell representation of Kinds., language=Haskell, label={lst:kinds}]
data Kind
    = KnStar
    | KnArr Kind Kind
    deriving (Eq, Show)
\end{lstlisting}

\subsection{Lifetime Checking}
The syntax of lifetime is show in Listing~\ref{lst:lifetimes}. \Note{explain
what each of the lifetime consturctors mean}

\begin{lstlisting}[caption=Haskell representation of Lifetimes.., language=Haskell, label={lst:lifetimes}]
data Lifetime 
    = LiVar String
    | LiLit Int
    | LiStatic
    | LiDummy
    deriving (Eq, Show)
\end{lstlisting}

\lstinline{LiDummy} is used here as a dummy place holder for occurrences of
lifetime literals during parsing, as lifetimes are associated with the scope of
terms. An initial walk of the tree replaces dummy values with the
\lstinline{LiLit} value that represents the depth of the scope that the
lifetime value is found in.

% \subsubsection{Ordering on Lifetimes}
% should probably write this, Ord instance for lifetimes.

\subsection{Type Checking}
Type checking is rather involved in this this language as the type system
almost contains the lambda calculus itself.

\begin{lstlisting}[caption=Haskell representation of Types., language=Haskell, label={lst:types}]
data Type
    = TyVar String
    | TyInt
    | TyArr Type Type
    | Forall String Kind Type
    | OpLam String Kind Type
    | OpApp Type Type
    deriving (Eq, Show)
\end{lstlisting}

\subsection{Error Reporting}
Any part of the program that may result in some kind of error is wrapped in a
partial application of Haskell's error monad, \lstinline{Except}, to a custom
error data type:

\begin{lstlisting}[caption=Partially applied error monad and language errors., language=Haskell, label={lst:types}]
type ThrowsError = Except LangErr

data LangErr
    = ParseError
    | NoMain
    | VarNotFound String
    | WrongKind Kind Kind
    | NotKnArr Kind
    | WrongType Type Type
    | NotTyArr Type
    | NotForall Type
    deriving (Eq, Show)
\end{lstlisting}

This has the advantage of being very composeable and also of reducing error
handling boilerplate in the program.

\subsection{Context Management}\label{sec:ctx}
The lifetime checker, kind checker, and type checker all rely on a variable
typing context. These contexts are threaded thought the program using Haskell's
environment monad, also known as the Reader monad. The record type holding
these contexts is shown in Listing~\ref{lst:contexts}. Contexts are represented
as a map from variable names are strings to some value, using Haskell's own
built in strict map data structure.

\begin{lstlisting}[caption=Record data type showing contexts., language=Haskell, label={lst:contexts}]
type Ctx = Map.Map String

data Env = Env
    { _typeCtx :: Ctx (Type, Lifetime)
    , _kindCtx :: Ctx Kind
    , _ltCtx :: Ctx Lifetime
    } deriving (Show, Eq)
\end{lstlisting}

The monad transformer stack where parsing, lifetime checking, kind checking, and type
checking take place is therefore:

\begin{lstlisting}[caption=Envirnment and error monad stack., language=Haskell, label={lst:monads}]
type Typing = ReaderT Env ThrowsError
\end{lstlisting}

\section{Testing}
The testing aspect of this project was very important, and was the only way to
have some idea that the program worked as expected. Specifically, these tests
needed to be automated so that any changes in the behaviour of the program could
be checked to make sure that they did not break any previous behaviours. Test
programs were manually written, and then loaded from file.

Testing code goes here, unit tests etc

The correctness of the parser, type checker, and evaluator was tested with the
HUnit testing framework for Haskell. This tool integrates with the stack build
tool so that all tests can be run with a single command. Tests themselves were
small programs-- these small programs were loaded from file, parsed, and then
type checked. The resulting type was then validated against the expected type
that was associated with each test file. This system had the advantage that
the whole process was checked any time that changes were made to the behaviour
of the type checker.

Certain tests were simply to test the parser aspect of the program but others
demonstrated key behavior expected from the langauge. These test are covered in
more deatil in Chapter~\ref{sec:testing}.

