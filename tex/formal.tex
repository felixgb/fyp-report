\chapter{Formal Definitions}\label{sec:formal}
This chapter attempts to formalize some of the concepts from Chapter~\ref{sec:motivation}.
The lambda calculus is introduced as a model of computation and as a system for
reasoning about features found in programming languages. Types are introduced
to the lambda calculus as well as other more complex extensions relevant to
this project. The higher order polymorphic lambda calculus, known as
System F $\omega$, is included as well as a lifetime--calculus devised for this
project.

\section{Lambda Calculus}\label{sec:lambdacalc}
The lambda calculus is model of computation where the only behaviour is
function definition and application. It is commonly used to specify programming
language features and to formalize their behaviour.
The reasons for useing the lambda calculus to specify programming behaviour are:
\begin{itemize}
    \item It is a very simple language, with very few constructs to reason about.
        This makes it very easy to implement an interpreter for, and very easy
        to implement type systems for.
    \item It is sufficiently powerful to express all computable functions.
        although it is extremely simple, it is a universal model of computation.
\end{itemize}
The lambda calculus can be view as a  miniature programming language or as a
system for modelling abstract behaviour. 
reason about type systems. The lambda calculus can be viewed as a miniature
programming language and a system where strict properties can be proved.



\begin{figure}[H]\label{fig:lambdaCalc}
    \begin{tabular}{l c p{3cm} r}
        \texttt{t} & $ ::= $ & $ x $ &                   variables \\
      & $ | $ & $ \lambda x $ . \texttt{t} &          function abstraction \\
      & $ | $ & \texttt{t} \texttt{t} &          function application \\
    \end{tabular}
    \caption{Grammer of the untyped lambda calculus.}
\end{figure}

\subsection{Simple Types in the Lambda Calculus}
The lambda calculus can be extended with the most simple of typing systems.
In order to demonstrate how the lambda calculus can be used to reason about
type systems, some additional constructs added. If Boolean literals some other
concepts are added it becomes more clear how types are useful in programming
languages.

\begin{figure}[H]\label{fig:simpleCalc}
    \vspace{1cm}
    \textbf{Terms} \\
    \begin{tabular}{l c p{3cm} r}
        \texttt{t} & $ ::= $ & $ x $ &                   variables \\
      & $ | $ & $ \lambda x : $ \texttt{T} . \texttt{t} &          function abstraction, with type annotations \\
      & $ | $ & $ true $ &          true literal \\
      & $ | $ & $ false $ &          false literal \\
      & $ | $ & if \texttt{t} then \texttt{t} else \texttt{t}  &          if-then-else expression \\
      & $ | $ & \texttt{t} \texttt{t} &          function application \\
    \end{tabular}

    \vspace{1cm}
    \textbf{Values} \\
    \begin{tabular}{l c p{3cm} r}
        \texttt{v} & $ ::= $ & $ \lambda x : $ \texttt{T} . \texttt{t} &                   lambda value \\
      & $ | $ & $ true $ &          true literal value \\
      & $ | $ & $ false $ &          false literal value \\
    \end{tabular}

    \vspace{1cm}
    \textbf{Types} \\
    \begin{tabular}{l c p{3cm} r}
        \texttt{T} & $ ::= $ & \texttt{T} $ \rightarrow $ \texttt{T} &      type of functions \\
      & $ | $ & $ Bool $ &          type of boolean literals \\
    \end{tabular}

    \caption{Lambda calculus extended with simple types and booleans.}
\end{figure}
Programs can be constructed using the language defined in Figure~\ref{fig:simpleCalc}
that do not make any sense. For example:

\begin{lstlisting}[mathescape, nolol, label={lst:simple}];
baddef = if $ \lambda x : Bool . x $ then $ false $  else $ true $
\end{lstlisting}

This program is constructed according to the grammar of terms. However, is not
well typed according to the typing rules in Figure~\ref{fig:simpleType}. \Note{make this diagram}
Specifically, the type of the term found in the guard of the if-expression,
\lstinline[mathescape]{$ \lambda x : Bool . x $} has type
\lstinline[mathescape]{$ Bool \rightarrow Bool $} but has to be
\lstinline[mathescape]{$ Bool $} for the term to be considered well typed.
It makes sense to restrict the sorts of values found in the guard of an if-then-else
expression. Anything other than a boolean value in that place may indicate
programmer error, which could be as simple as a spelling mistake or a fundamental
misunderstanding of what they are trying to express. Type systems help catch
these kind of mistakes before program execution. Specifying these rules on top
of the lambda calculus can be a very informative way of looking at programming
language features. The rest of this chapter describes other extensions to the
lambda calculus that model some of the programming concepts investigated in
this report.

\subsection{System F}\label{sec:systemF}
System F is an extension of the simply typed lambda calculus \cite{tapl} that
allows for quantification over types as well as terms. In doing this it
formalizes the notion of polymorphism in programming languages, as described in
Section~\ref{sec:generics}. It is used to study implementations of polymorphism
in programming languages.

As the simply typed lambda calculus allows for abstraction of terms outside of
terms through function definitions, System F introduces abstractions at the
level of types. The system also allows for application of type level
expressions. This system can be used to reason about first order polymorphism,
however more extensions to talk about higher order polymorphism. These extensions
are introduced in Section~\ref{sec:omega}.

\begin{figure}[H]\label{fig:systemF}
    \vspace{1cm}
    \textbf{Terms} \\
    \begin{tabular}{l c p{3cm} r}
        \texttt{t} & $ ::= $ & $ x $ &                   variables \\
      & $ | $ & $ \lambda x : $ \texttt{T} . \texttt{t} &          function abstraction, with type annotations \\
      & $ | $ & \texttt{t} \texttt{t} &          function application \\
      & $ | $ & $ \lambda X $ . \texttt{t} &          type abstraction \\
      & $ | $ & \texttt{t} [\texttt{T}] &          type application \\
    \end{tabular}

    \vspace{1cm}
    \textbf{Values} \\
    \begin{tabular}{l c p{3cm} r}
        \texttt{v} & $ ::= $ & $ \lambda x : \texttt{T} $ . \texttt{t} &                   lambda value \\
      & $ | $ & $ \lambda X . $ \texttt{t} &   type abstraction value \\
    \end{tabular}

    \vspace{1cm}
    \textbf{Types} \\
    \begin{tabular}{l c p{3cm} r}
        \texttt{T} & $ ::= $ & \texttt{T} $ \rightarrow $ \texttt{T} &      type of functions \\
        & $ | $ & $ X $ &       type variables \\
        & $ | $ & $ \forall X . $ \texttt{T} &       universal type \\
    \end{tabular}

    \caption{System F}
\end{figure}

\subsubsection{Examples}
System F models first order polymorphism in programming languages, such as aspect of
the generics found in Java. Some examples of how System F can be used are given
below:

\begin{lstlisting}[mathescape, nolol, label={lst:systemf}];
id = $ \lambda X . \lambda x: X . x $
boolId = id [Bool]
\end{lstlisting}

This example shows type level lambdas, where $ X $
is a type variable that has been abstracted out of the function definition. The
term level lambda abstracts $ x $ which has the type $ X $. Applying \lstinline{id}
to a type maybe be written as \lstinline[mathescape]{id [Bool]}, which 
yields \lstinline[mathescape]{$ \lambda x: Bool . x $}, or the identify function
over boolean values. Here $ X $ in the original definition of \lstinline{id} has
been instantiated with the type of Booleans.

\subsection{System F $\omega$}\label{sec:omega}
System F $\omega$ is also known as the higher order polymorphic lambda calculus. It
takes concepts from the polymorphic lambda calculus, or System F, and combines
these with type constructors described in Section~\ref{sec:tyops}. From a
programming point of view, System F~$\omega$ allows us to write polymorphic
functions not only at the level of types, but also in type constructors.

Most usefully System F $\omega$ can be used to build type safe abstractions
around traditionally impure forms of computation; typically in the form of
a \textit{monad}. Not only is higher order polymorphism very good at formalizing
monads, but also very good at combining them. \textit{Monad transformers} are
a key example, but outside the scope of this paper. An example of monad transformers
in use can be found in Section~\ref{sec:ctx}. Haskell uses the concepts to
great effect.

\subsubsection{Type Operators}\label{sec:tyops}
Type constructors were mentioned in Section~\ref{sec:generics} and this
section attempts to frame this idea within the lambda calculus. As mentioned
previously, type constructors can be thought of functions from types to new
types. Application and abstraction at the level of types can be modelled as a
type system for the lambda calculus.            

\begin{figure}[H]\label{fig:tyops}
    \vspace{1cm}
    \textbf{Terms} \\
    \begin{tabular}{l c p{3cm} r}
        \texttt{t} & $ ::= $ & $ x $ &                   variables \\
      & $ | $ & $ \lambda x : $ \texttt{T} . \texttt{t} &          function abstraction, with type annotations \\
      & $ | $ & \texttt{t} \texttt{t} &          function application \\
    \end{tabular}

    \vspace{1cm}
    \textbf{Values} \\
    \begin{tabular}{l c p{3cm} r}
        \texttt{v} & $ ::= $ & $ \lambda x : \texttt{T} $ . \texttt{t} &                   lambda value \\
    \end{tabular}

    \vspace{1cm}
    \textbf{Types} \\
    \begin{tabular}{l c p{3cm} r}
        \texttt{T} & $ ::= $ & \texttt{T} $ \rightarrow $ \texttt{T} &      type of functions \\
        & $ | $ & $ \lambda X :: \texttt{K} . \texttt{T} $ &            operator lambda \\
        & $ | $ & \texttt{T} \texttt{T} &                           operator application \\
        & $ | $ & $ X $ &       type variables \\
    \end{tabular}

    \vspace{1cm}
    \textbf{Kinds} \\
    \begin{tabular}{l c p{3cm} r}
        \texttt{K} & $ ::= $ & $ * $ &      kind of concrete types \\
        & $ | $ & \texttt{K} $ \rightarrow $ \texttt{K} &            kind of type operators
    \end{tabular}

    \caption{Type operators and Kinding}
\end{figure}

Kinds play a key role here. Just as types enforce well-formed expressions at
the level of terms, Kinds are needed to check that expressions at the level of
types are well formed. In other words, Kinds do not allow non-sensicle expressions
like \lstinline{Bool Int}, because Bool is not a type constrcutor. Concrete
types, also called base types, have Kind \lstinline{*}. These are the types
that may be the type of terms in the lanaguage like \lstinline[mathescape]{false}.
Type constructors, like \lstinline{List} have kind \lstinline[mathescape]{* \rightarrow *}.
They need to be parametrized with a type that had kind \lstinline{*}.

Type operators combined with System F forms the basis of System F $\omega$.
In effect System F $\omega$ turns type constructors into first class values of
the language. Kind annotations are carried over from the lambda calculus with type
constructors.

\begin{figure}[H]\label{fig:systemfomega}
    \vspace{1cm}
    \textbf{Terms} \\
    \begin{tabular}{l c p{3cm} r}
        \texttt{t} & $ ::= $ & $ x $ &                   variables \\
      & $ | $ & $ \lambda x : $ \texttt{T} . \texttt{t} &          function abstraction, with type annotations \\
      & $ | $ & \texttt{t} \texttt{t} &          function application \\
      & $ | $ & $ \lambda X :: $ \texttt{K} . \texttt{t} &          type abstraction \\
      & $ | $ & \texttt{t} [\texttt{T}] &          type application \\
    \end{tabular}

    \vspace{1cm}
    \textbf{Values} \\
    \begin{tabular}{l c p{3cm} r}
        \texttt{v} & $ ::= $ & $ \lambda x : \texttt{T} $ . \texttt{t} &                   lambda value \\
         & $ | $ & $ \lambda X :: \texttt{K} $ . \texttt{t} &                   type abstraction value \\
    \end{tabular}

    \vspace{1cm}
    \textbf{Types} \\
    \begin{tabular}{l c p{3cm} r}
        \texttt{T} & $ ::= $ & \texttt{T} $ \rightarrow $ \texttt{T} &      type of functions \\
        & $ | $ & $ \lambda X :: \texttt{K} . \texttt{T} $ &            operator lambda \\
        & $ | $ & \texttt{T} \texttt{T} &                           operator application \\
        & $ | $ & $ X $ &       type variables \\
        & $ | $ & $ \forall X :: $ \texttt{K} . \texttt{t} &       universal type \\
    \end{tabular}

    \vspace{1cm}
    \textbf{Kinds} \\
    \begin{tabular}{l c p{3cm} r}
        \texttt{K} & $ ::= $ & $ * $ &      kind of concrete types \\
        & $ | $ & \texttt{K} $ \rightarrow $ \texttt{K} &            kind of type operators
    \end{tabular}

    \caption{System F $\omega$}
\end{figure}

\subsubsection{Examples}
examples of programming in System F $\omega$

\section{Lifetimes}
syntax of lifetimes

\subsection{Adding Impurity}
var references, borrow types, etc

\section{Final Typing Rules}
final grammar and typing derivations for the language

